% ******************************************************************************
% ****************************** Custom Margin *********************************

% Add `custommargin' in the document class options to use this section
% Set {innerside margin / outerside margin / topmargin / bottom margin}  and
% other page dimensions
\ifsetCustomMargin
  \RequirePackage[left=37mm,right=30mm,top=35mm,bottom=30mm]{geometry}
  \setFancyHdr % To apply fancy header after geometry package is loaded
\fi

% Add spaces between paragraphs
%\setlength{\parskip}{0.5em}
% Ragged bottom avoids extra whitespaces between paragraphs
\raggedbottom
% To remove the excess top spacing for enumeration, list and description
%\usepackage{enumitem}
%\setlist[enumerate,itemize,description]{topsep=0em}

% *****************************************************************************
% ******************* Fonts (like different typewriter fonts etc.)*************

% Add `customfont' in the document class option to use this section

\ifsetCustomFont
  % Set your custom font here and use `customfont' in options. Leave empty to
  % load computer modern font (default LaTeX font).
  %\RequirePackage{helvet}

  % For use with XeLaTeX
  %  \setmainfont[
  %    Path              = ./libertine/opentype/,
  %    Extension         = .otf,
  %    UprightFont = LinLibertine_R,
  %    BoldFont = LinLibertine_RZ, % Linux Libertine O Regular Semibold
  %    ItalicFont = LinLibertine_RI,
  %    BoldItalicFont = LinLibertine_RZI, % Linux Libertine O Regular Semibold Italic
  %  ]
  %  {libertine}
  %  % load font from system font
  %  \newfontfamily\libertinesystemfont{Linux Libertine O}
\fi

% *****************************************************************************
% **************************** Custom Packages ********************************

% ************************* Algorithms and Pseudocode **************************

%\usepackage{algpseudocode}


% ********************Captions and Hyperreferencing / URL **********************
%my edit
\usepackage{hyperref}
\hypersetup{urlcolor=blue, colorlinks=true}  % Colours hyperlinks in blue, but this can be distracting if there are many links.


% Captions: This makes captions of figures use a boldfaced small font.
\RequirePackage[small,bf]{caption}

\RequirePackage[labelsep=space,tableposition=top]{caption}
\renewcommand{\figurename}{Fig.} %to support older versions of captions.sty


% *************************** Graphics and figures *****************************

\usepackage{rotating}
\usepackage{wrapfig}

% Uncomment the following two lines to force Latex to place the figure.
% Use [H] when including graphics. Note 'H' instead of 'h'
%\usepackage{float}
%\restylefloat{figure}

% Subcaption package is also available in the sty folder you can use that by
% uncommenting the following line
% This is for people stuck with older versions of texlive
%\usepackage{sty/caption/subcaption}
\usepackage{subcaption}

% ********************************** Tables ************************************
\usepackage{booktabs} % For professional looking tables
\usepackage{multirow}
\renewcommand{\arraystretch}{1.2}
%\usepackage{multicol}
%\usepackage{longtable}
%\usepackage{tabularx}


% *********************************** SI Units *********************************
\usepackage{siunitx} % use this package module for SI units


% ******************************* Line Spacing *********************************

% Choose linespacing as appropriate. Default is one-half line spacing as per the
% University guidelines

% \doublespacing
% \onehalfspacing
% \singlespacing


% ************************ Formatting / Footnote *******************************

% Don't break enumeration (etc.) across pages in an ugly manner (default 10000)
%\clubpenalty=500
%\widowpenalty=500

%\usepackage[perpage]{footmisc} %Range of footnote options


%%%%%%%%%%%%%%%%%%%%%%%%%%%%%%%%%%%%
% Make some personal commands
%%%%%%%%%%%%%%%%%%%%%%%%%%%%%%%%%%%%
\usepackage{xspace}

%Decays
\newcommand{\bsmumu}{$B_{s}^{0} \to \mu^{+} \mu^{-}$\xspace}
\newcommand{\bdmumu}{$B^{0} \to \mu^{+} \mu^{-}$\xspace}
\newcommand{\bmumu}{$B^{0}_{(s)} \to \mu^{+} \mu^{-}$\xspace}
\newcommand{\bhh}{$B \to h^{+} h^{'-}$\xspace}
\newcommand{\bdkpi}{$B^{0} \to K^{+} \pi^{-}$\xspace}
\newcommand{\bskk}{$B^{0}_{s} \to K^{+} K^{-}$\xspace}
\newcommand{\bskpi}{$B^{0}_{s} \to K^{+} \pi^{-}$\xspace}
\newcommand{\bdpipi}{$B^{0} \to \pi^{+} \pi^{-}$\xspace}
\newcommand{\bsjpsiphi}{$B^{0}_{s} \to J/\psi \phi$\xspace}
\newcommand{\bujpsik}{$B^{+} \to J/\psi K^{+}$\xspace}

\newcommand{\lambdab}{$\Lambda^{0}_{b}\to p \mu^{-} \nu_{\mu}$\xspace}
\newcommand{\bdpimunu}{$B^{0} \to \pi^{-} \mu^{+} \nu_{\mu}$\xspace}
\newcommand{\bsKmunu}{$B^{0}_{s} \to K^{-} \mu^{+} \nu_{\mu}$\xspace}
\newcommand{\bdpimumu}{$B^{0} \to \pi^{0} \mu^{+} \mu^{-}$\xspace}
\newcommand{\bupimumu}{$B^{+} \to \pi^{+} \mu^{+} \mu^{-}$\xspace}
\newcommand{\bpimumu}{$B^{0(+)} \to \pi^{0(+)} \mu^{+} \mu^{-}$\xspace}
\newcommand{\bcjpsimunu}{$B^{+}_{c} \to J/\psi \mu^{+} \nu_{\mu}$\xspace} 
\newcommand{\jpsimumu}{$J/\psi \to \mu^{+} \mu^{-}$\xspace}
\newcommand{\bbbarmumux}{$b\bar{b} \to \mu^{+} \mu^{-} X$\xspace}
\newcommand{\bbbar}{$b\bar{b}$\xspace}
\newcommand{\bsd}{$B_{(s)}^{0}$\xspace}
\newcommand{\bs}{$B_{s}^{0}$\xspace}
\newcommand{\bd}{$B^{0}$\xspace}
\newcommand{\pdfs}{PDFs\xspace}
\newcommand{\pdf}{PDF\xspace}

\newcommand{\elm}{effective lifetime measurement\xspace}
\newcommand{\el}{effective lifetime\xspace}
\newcommand{\BF}{branching fraction\xspace}
\newcommand{\BFm}{branching fraction measurements\xspace}

\newcommand{\jpsi}{$J/\psi$\xspace}
\newcommand{\chisqd}{$\chi^{2}$\xspace}
%Lifetime terms
\newcommand{\ADG}{$A_{\Delta \Gamma}$\xspace}
\newcommand{\tmumu}{$\tau_{\mu \mu}$\xspace}
\newcommand{\BFs}{branching fractions\xspace}
\newcommand{\invtmumu}{$\tau^{-1}_{\mu \mu}$\xspace}
\newcommand{\lt}{$\tau$\xspace}
\newcommand{\invlt}{$\tau^{-1}$\xspace}
\newcommand{\tH}{$\tau_{H}$\xspace}
\newcommand{\tL}{$\tau_{L}$\xspace}
\newcommand{\Gmumu}{$\Gamma_{\mu \mu}$\xspace}

\newcommand{\ml}{maximum likelihood\xspace}

%Units
\newcommand{\tev}{TeV\xspace}
\newcommand{\gev}{GeV\xspace}
\newcommand{\mev}{MeV\xspace}
\newcommand{\tevcc}{TeV/$c^{2}$\xspace}
\newcommand{\gevcc}{GeV/$c^{2}$\xspace}
\newcommand{\mevcc}{MeV/$c^{2}$\xspace}
\newcommand{\tevc}{TeV/$c$\xspace}
\newcommand{\gevc}{GeV/$c$\xspace}
\newcommand{\mevc}{MeV/$c$\xspace}
\newcommand{\fb}{fb$^{-1}$\xspace}
\newcommand{\pb}{pb$^{-1}$\xspace}
\newcommand{\invps}{ps$^{-1}$\xspace}
\newcommand{\bhadron}{$b$-hadron\xspace}
\newcommand{\ps}{ps\xspace}
\newcommand{\bhadrons}{$b$-hadrons\xspace}

%Kinimatic stuff
\newcommand{\p}{$p$\xspace}
\newcommand{\pt}{$p_{T}$\xspace}

% *****************************************************************************
% *************************** Bibliography  and References ********************

%\usepackage{cleveref} %Referencing without need to explicitly state fig /table

% Add `custombib' in the document class option to use this section
\ifuseCustomBib
   \RequirePackage[square, sort, numbers, authoryear]{natbib} % CustomBib

% If you would like to use biblatex for your reference management, as opposed to the default `natbibpackage` pass the option `custombib` in the document class. Comment out the previous line to make sure you don't load the natbib package. Uncomment the following lines and specify the location of references.bib file

%\RequirePackage[backend=biber, style=numeric-comp, citestyle=numeric, sorting=nty, natbib=true]{biblatex}
%\bibliography{References/references} %Location of references.bib only for biblatex

\fi

% changes the default name `Bibliography` -> `References'
\renewcommand{\bibname}{Bibliography}

% ******************************** Roman Pages *********************************
% The romanpages environment set the page numbering to lowercase roman one
% for the contents and figures lists. It also resets
% page-numbering for the remainder of the dissertation (arabic, starting at 1).

\newenvironment{romanpages}{
  \setcounter{page}{1}
  \renewcommand{\thepage}{\roman{page}}}
{\newpage\renewcommand{\thepage}{\arabic{page}}}


% ******************************************************************************
% ************************* User Defined Commands ******************************
% ******************************************************************************

% *********** To change the name of Table of Contents / LOF and LOT ************

%\renewcommand{\contentsname}{My Table of Contents}
%\renewcommand{\listfigurename}{My List of Figures}
%\renewcommand{\listtablename}{My List of Tables}


% ********************** TOC depth and numbering depth *************************
% I think this changes the titles
\setcounter{secnumdepth}{3}
\setcounter{tocdepth}{2}
%\subsection*{Introduction} %This gives an example of how to get an un-numbered heading that does not go into the table of contents

% ******************************* Nomenclature *********************************

% To change the name of the Nomenclature section, uncomment the following line

%\renewcommand{\nomname}{Symbols}


% ********************************* Appendix ***********************************

% The default value of both \appendixtocname and \appendixpagename is `Appendices'. These names can all be changed via:

%\renewcommand{\appendixtocname}{List of appendices}
%\renewcommand{\appendixname}{Appndx}

% *********************** Configure Draft Mode **********************************

% Uncomment to disable figures in `draftmode'
%\setkeys{Gin}{draft=true}  % set draft to false to enable figures in `draft'

% These options are active only during the draft mode
% Default text is "Draft"
%\SetDraftText{DRAFT}

% Default Watermark location is top. Location (top/bottom)
%\SetDraftWMPosition{bottom}

% Draft Version - default is v1.0
%\SetDraftVersion{v1.1}

% Draft Text grayscale value (should be between 0-black and 1-white)
% Default value is 0.75
%\SetDraftGrayScale{0.8}


% ******************************** Todo Notes **********************************
%% Uncomment the following lines to have todonotes.

%\ifsetDraft
%	\usepackage[colorinlistoftodos]{todonotes}
%	\newcommand{\mynote}[1]{\todo[author=kks32,size=\small,inline,color=green!40]{#1}}
%\else
%	\newcommand{\mynote}[1]{}
%	\newcommand{\listoftodos}{}
%\fi

% Example todo: \mynote{Hey! I have a note}

%An attempt to change the titles
%  \RequirePackage{titlesec}
%        \newcommand{\PreContentTitleFormat}{\titleformat{\chapter}[display]{\scshape\Large}
%        {\Large\filleft{\chaptertitlename} \Huge\thechapter}
%        {1ex}{}
%        [\vspace{1ex}\titlerule]}
%        \newcommand{\ContentTitleFormat}{\titleformat{\chapter}[display]{\scshape\huge}
%        {\Large\filleft{\chaptertitlename} \Huge\thechapter}{1ex}
%        {\titlerule\vspace{1ex}\filright}
%        [\vspace{1ex}\titlerule]}
%        \newcommand{\PostContentTitleFormat}{\PreContentTitleFormat}
%        \PreContentTitleFormat



%Things that I have added to try and get things to work.

\usepackage{afterpage}
