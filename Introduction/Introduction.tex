\chapter{Introduction}

The Standard Model (SM) of particle physics is a Quantum Field Theory that attempts to describe the building blocks of matter and their interactions at a fundemental level. It has been built up using a combination of progress in theortical physics and experiment discoveries. 
The SM predicts a set of fermions that matter can be made from and bosons that govern the interactions between the them. There are two types of fermions quarks ({\it u, d, s, c, t, b}) and leptons ($e^-, \nu_e, \mu^{-}, \nu_\mu, \tau^{-}, \nu_\tau$) each of which has a anti-particle with the same mass but oposite charge. There are three forces that describe the interactions between these fermions; the strong, weak and electromagnetic forces. Each force associated with a gauge boson ($g, W^\pm, Z^0, \gamma$) that mediates the forces between particles. The final particle in the SM is the Higgs boson ($H^0$) and the interactions of all particles with the field associated with this boson is what is responsible for the intrinsic mass of each particle.
%Do I want to mention anything about mesons and baryons and also that the everyday matter is only made up of a few of these and also the charges of the particles?

The SM is used to produce predictions of particle decays and interatctions which have been compared to experimental results over the past decades. So far the SM has held up well to the different tests it has been put under and no significant deviations of processes from the SM predictions have been found. All fundemental particles included within the SM framework have been found, the latest of which was the Higgs boson which was discovered by the ATLAS and CMS experiments in 2012 at the Large Hadron Collider. 

Although the SM is very sucessful there are several reasons to keep testing the model. Firstly the are observed phenomona that are not included within the scope of the SM. % and there are also more asthetic questions as to why the SM has the properties that is does. 
In its current for the SM cannot explain the observed oscillation of neutrinos from one type into another~\cite{} and it does not provide a particle or mechanism that could account of the observed presence of dark matter and energy in the universe~\cite{}. At the start of the universe matter and anti-matter should have been produced in eqaul amount but that is not what is observed in the unierse today and the SM does not include a mechanism large enough to account to this asymetry and the gravitational force is not included in the theory.
%Do I want to put more on this? Or prehaps I could point the reader to somewhere that explains more?
Secondly there are more asthetic questions as to why the SM has the properties that is does. 
The SM provides no reason to explain the huge difference in the coupling strenghts of the different forces or account for the number of fermions and why the masses are very differet, for example to top quark mass is X and the up quark mass is Y. The observed mass of the Higgs boson is very low compared to ... which is unsatisfying. 
The questions posed here that the SM cannot answer are some of the problems with the SM, a more complete discussion can be found in~\cite{}. 
The questions asked about the SM indicated that it is a low energy effective theory with approximatly described a more fundemental theory and it is the search for what this theory could be that motivated the contiuning study of the SM. 

There exist many theories that go beyond to scope of the SM and seek to explain what the SM cannot. Consequently, these theories predict the presence of new particles and phemonona that are collectively called New Physics (NP). 
Thereforeo it important to precisely test the predictions of the SM to discover where is works and to search for NP effects. The Large Hadron Collider is one such places where these tests are performed through the study of high energy proton-proton, $pp$, collisions. There are two approaches used to search for NP effects in collider experiments, direct and indirect searchs for NP effects.

Direct searches involve looking specifically for NP particles and phemona that could be produced in high energy collisiosn, similar to the the search for the Higgs boson. This type of search is limited by the energy avalible in collision as to the mass of a new particle that could be produced. The observation of something new would be conclusive proof of new physics and the lack of observation constrains the parameter space of NP theories.

Indirect searches aim to precisely measure SM processes and look for deviateion in the measured values from the predicted values. Deviations can be caused by the presence of NP effects that modify the SM process. Indirect searches can reveal the presence of NP effect or in the absence of NP constraints can be places on the models.
%There is a lot of repetition! :(
Process that occur rarely within the SM offer excellent platforms to search for NP effect because the NP contorbution can be of the same order of magnitude as the SM value. %I could add more here about how NP enters in but I don't think that is relevant here.
One such type of decays are \bmumu decays. The fraction of \bs and \bd decays decaying into two muons is very small at $\sim 10^{-9}$ and $\sim 10^{-10}$, respectively. It was hoped that these decays could sho a significant deviation from the SM prediction. These decays have been studies for over 30 years with increasingly sensitive searches, as shown in Figure~\ref{}. The latest experiments to stufdy are the ATLAS, CMS and LHCb experiment on the Large Hadron Collider. The unpredicented center-of-mass at the LHC enables greatest sensitivity to these decays than any past experiment.  
The data collected from 2010 to 2012 from $pp$ collisions at the LHC with center of mass energies at 7 and 8~\tev
%Run~1 of the LHC from from 2010 to 2012%
 produced the first observation of \bsmumu decays and the first evidence of \bdmumu with the combined data set of the CMS and LHCb collaborataions. The measured branching fractions were measured as


with a statistical significance of 6.0 $\sigma$ for the \bs mode and 3.0 $\sigma$ for the \bd. Similarly the ATLAS experiment searched for \bmumu decay using data collected during the same period and measured the \BF as

with a statistical signigicance of X $\sigma$ for the \bs mode and Y $\sigma$ for the\bd, therefore a limit was place of the \bd \BF at YYY. Although it was hoped that significant deviations from the SM predictions would be found in these decays this was not the case. All results are consistent with thw SM values, with X standard deviations, and have enalbed constraints to be place on the parameter space of new physics models~\cite{}. However the precision of the measurements still allows plenty of space for NP effect to be revealed and therefore the study is these decays is still very relevant. The observation of \bsmumu decays allows other parameters of the decay to be measured, this includes the effective lifetime which complementary to the \BF measurement in the search for NP effects. A comparison of the measured \BF values and the SM predictions is shown in Figure~\cite{}.

At the moment the SM is holding up well under the test that have been preformed. Direct seraches have revealed no NP particles but there is some tension between measured results and SM prediction in indirect searches for NP through rare B-meson decays. Hints at disgreements with the SM have been seen in the $b \to a sll$ transitions through the angular distribution of $B^0 \to K^{*0} \mu^{+} \mu^{-}$ decays, the branching fraction of $B^{0}_{s} \to \phi  \mu^{+} \mu^{-}$ decays and the ratio $R(K)$ which is the ratio of the branching fraction of $B^+ \to K^+ \mu^{+} \mu^{-}$ and $B^+ \to K^+ e{+} e^{-}$. Also the ratios $R(D)$ and $R(D^*)$ of the branching fractions of $B^0 \to D^{(*)} \tau^{-} \nu_{\tau}$ and $B^0 \to D^{(*)} \mu^{-} \nu_{\mu}$ provide hints of NP processes. These anomalies have been seen in several different experiments and when conbining the result for $B^0 \to K^{*0} \mu^{+} \mu^{-}$ and $R(D^{(*)})$ from LHCb, BarBar and Belle, the combined results are 4 $\sigma$ from the SM predictions.


This dissertation documents to study of \bmumu decays at the LHCb experiment using data collected during $pp$ collisions at a center-of-mass energy of 7, 8 and 13~\tev. The LHC and LHCb experiment are describedi in Chapter~\ref{} and the theoritical motivation for studying \bmumu decays is given in Chapter~\ref{}. Both the \BF of these decays and the effective lifetime of \bsmumu decays offer completmentry observables to test the SM and search for NP effects. The selection criteria used to identify these decays in the data collected by the LHCb experiment are detailed in Chapter~\ref{} and the measurement of the \BF is brielfy covered in Chapter~\ref{}. The measurement of the effective lifetime is detail in Chapter~\ref{} and the systematic uncertainies on this measurement are given in Chapter~\ref{}. Finally summary of the results and prospects for future measurements of \bmumu decays are given in Chapter~\ref{}.


%Although the SM has been proven to be an exteremly sucessful theory there are reasons to keep testing it. These reasons fall into two main categories; the completness of the SM and the naturatlness. The completeness of the SM describes that there are observed phemonnona that are not included in the framework of the SM. In its current for the SM cannot explain the observed oscillation of neutrinos from one type into another~\cite{} and it does not provide a particle or mechanism that could account of the observed presence of dark matter and energy in the universe~\cite{}. At the start of the universe matter and anti-matte
