\chapter{Selection of \bsmumu events for the effective lifetime Measurement}
\label{selection_chapter}

This chapter explains how \bsmumu decays are selected from the data collected at the LHCb experiment. There are various steps to the selection process than enable a relatively clean sample of \bsmumu decays to be selected to be used in the analysis described in Chapter X. Although the muons from \bsmumu decays leave a clear, easy to identify siganture in the LHCb detector muons in particle decays and interactions as well as hadrons that are mis-identified as muons can be combined to mimic \bsmumu decay. These other processes occur more often than the very rare \bsmumu decays making identifying the actual signal challenging. The selection is designed to remove as many background events as possible whilst keeping a high signal efficiency. The details of the processing that lead to the most imortant background events are outlined in Section \ref{sec:backgroundoutline}.

 The development selection relies on Monte Carlo simulated events, those needed for this selection are detailed in Section \ref{sec:MCsamples}. 

The selection of \bsmumu decays proceeds in several steps, firstly trigger requirements are used and then there is a stripping selection. The we get some more stuff.

The selection of \bsmumu decays has been developed over many years by members of the LHCb collaborations for the measurement of the \bsmumu branching fraction. Not all parts of this Chapter are my work, but I have worked on the areas ...

%I've not explained what a candidate is - should I?

\subsection{Backgrounds}
\label{sec:backgroundoutline}


\subsection{Monte Carlo Samples}
\label{sec:MCsamples}


\subsection{Trigger}
\label{sec:triggerRequirements}

\subsection{Stripping}
\label{sec:stripping}


\subsection{Pre-selection/Offline selection?}

\subsection{Particle Identification}

\subsection{Multivariate Classifier}
