\chapter{Selection of \bsmumu events for the effective lifetime Measurement}
\label{selection_chapter}
The analysis described in Chapter X for the \bsmumu effective lifetime requires \bsmumu and \bhh decays to be identified in the data sets recorded by the LHCb experiment. Although \bsmumu decays leave a clear 2 muon signature in the detector, the selection of these decays is challenging because it is a very rare process and there are many other processes that can mimic a \BsMuMu decay in the detector. The background processes are described in Section~\ref{sec:backgroundoutline}. To understand different aspects of the selection and analysis of \BsMuMu decays, particle decays with a similar topology to \BsMuMu are used. \bhh decays, where $h = K, \pi$, are used because they have large branching fractions and are well understood from pervious LHCb analyses as well as a similar topology to \bsmumu decays. 
%The selection of \bhh decays is kept as close as possible to the selection of \bsmumu so they can be used as a validation channels.
The measurement of the \bmumu Branching Fractions, described in Chapter X, requires the use of \bujpsik decays, as well as \bmumu and \bhh decays, to be used as a noramlisation channel.
This Chapter describes the selection of \bmumu, \bhh and \bujpsik decays for the effective lifetime and Branching fraction analyses. The analyses share many of the same selection requirements. The selection occurs in several stages, and the development of the selection relies on simulated events which are detailed in Section~\ref{sec:MCsamples}. The first step to select decays is choosing what requirements to place on the trigger which is followed by a set of loose selection requirements to remove obvious background events. These two steps are described in Sections~\ref{sec:triggerRequirements} and \ref{sec:stripping}. A tighter selection is applied to the output of the stripping as described in Section~\ref{sec:offline_sel} and particle identification requirements are used in Section~\ref{sec:PID} to further reduced background events. Finally a multivariate classifier is described in Section~\ref{sec:BDT} used as the final step in the selection to reduced the backgrounds to a level suitable for the analysis in Chapter X to be preformed. Throughout this Chapter \bsmumu and \bdmumu are selected in the same way.

The LHCb collaboration has published a number of papers studying the \bsmumu decay, the selection described in this Chapter has been built up over a number of years by a range of different collaboration members. The studies detailed in sections X, X and Y was done for this thesis.





\section{Backgrounds}
\label{sec:backgroundoutline}
%A \bs decaying into two muons leaves information in the LHCb detector with certain identifying characteristics. The two muons form a good vertex that is displaced from the primary vertex of the event because the Bs has a long lifetime and the combined momentum of the muons can be extrapolated backwards to the primary vertex because the muons are the only decay products of the \bs. There are other processes that occur in proton-proton decays that can leave information in the detector in a similar pattern to \bsmumu decays. The reconstruction, described in Section X, produces many \bsmumu candidates, the aim of the selection is to separate the real \bsmumu decays from the background in the reconstructed candidates.

%The background sources for \bsmumu decays can be split into two groups, those that have quite obvious difference from \bsmumu decays and those that do not. The first set can be removed from the data set by taking advantage of the obvious differences whilst keeping a high 

The reconstuction process, outlined in Section X, produces numerous \bsmumu candidates from pairs of muons created by $pp$ collisions and recorded in the detector. Some candidates will have come from real \bsmumu decays but there are other processes that occur during $pp$ collisions that can create two muons that when conbined look a lot like a \bsmumu decay. %In the detector a \bsmumu decay will produce two muons that form a good vertex which is displaced from the primary vertex where the \bs was produced.   

%The selection aims to seperate real \bsmumu decyas from these background to produce a set of \bsmumu candidates with a high signal purity from which the \bs effective lifetime can be measured. (here?)
%The main sources of background processes for \bsmumu decays are;
The main sources of background decays that mimic \bsmumu decays are:
\begin{itemize}
\item Elastic collisions of protons that produce a pair of muons via the exhange of a photon, $pp \to p \mu^{+} \mu^{-} p$. The proton are travel down the beam pipe and are undetected leaving the muons to be reconstructed as \bsmumu. Typically the muons produced in this way have low transverse momentun. %whilst the protons travel down the beam pipe. The muons produced have low transverse momentum.
\item Inelastic proton collisions that create two muons at the primary vertex. The muons for a good vertex and are combined to for a \bs that decays instantaneously. This type of background is prompt combinatorial background. 
\item $B_{}^{0}\to\mu^{+}\mu^{-}\gamma$ decays where the photon is not reconstructed. The presence of the photon in the decay means that $B_{S}^{0}\to\mu^{+}\mu^{-}\gamma$ decays are not helicity suppressed and could therefore be a sizeable background, however the photon gains a large transverse momentum resulting in the reconstructed \bs mass being much lower than expected.
\item Random combinations of muons produced by seperate semi-leptonic decays. The \bsmumu candidates formed in this way are long lived combinatorial background because the reconstruted \bs will not decay instantansourly. %can be formed when muons produced in separated semi-leptonic decays are combined. These are known as long lived combinatorial background because the reconstructed \bs will not decay instantaneously.
\item Semi-leptonic decays where one of the decay products is mis-identified as a muon and/or is not detected. The resulting mass of the \bs candidate is lower than expected due to the missing particle information. The semi-leptonic decays that contribute to \bsmumu backgrounds in this way are \bdpimunu, \bsKmunu, \bpimumu, \bdpimumu and \bcjpsimunu where \jpsimumu.
\item \bhh decays, where $ h  = K, \pi$, where both hadrons are mis-identified as muons. This usually occurs when the hadrons decay in flight. Similary to mis-identified semi-leptonic decays the reconstructed \bs candidate mass is lower than expected.
\item \bdmumu decays that are identical to \bsmumu decays apart from the difference in the $B$ meson masses. The \bd decay is irrelevant for the measurement of the \bsmumu effective lifetime and is therefore a background for this measurement.
\end{itemize}

The selection aims to seperate real \bsmumu decyas from the background to produce a set of \bsmumu candidates with a high signal purity from which the \bs effective lifetime can be measured. This is challenging because \bsmumu decays are highly suppressed decays therefore reconstructed candidates are predominately made from background decays.

%The \bdmumu is also a background process for measuring the \bsmumu effective lifetime, the only way to seperate \bsmumu and \bdmumu decays is by using the different masses of the \bs and $B^{0}$ mesons. (In the bullet points?)

%Maybe say how since the decay is so rare there are many many more reconstruced background decays than real bsmumu decays?

{\it I could put a plot showing the mass plot from the previous analysis or I could make a plot something like Siim has to illustrate what i mean but that feels a bit like copying!}
%Separating the backgrounds from the \bsmumu decays can be done relatively straightly forwardly for many of the background processes by taking advantage of the obvious differences between the background and \bsmumu decays. However, distinguishing \bsmumu decays from long lived combinatorial backgrounds, and mis-idetificed \bhh and semi-leptonic decays is more challenging and \bsmumu decays must be sacrificed in order to remove a sufficient about of the background processes for the analysis to be performed. For the effective lifetime analysis the \bdmumu decay is not relevant and is therefore a background, however since the decays are extremely similar the \bsd masses are the only way to seperate the decays.

\section{Simulated Particle Decays}
\label{sec:MCsamples}
Simulated particle decays, as described in Section X, are used to develop the selection and analysis of \bsmumu decays. Large clean samples of simulated decays are needed to seperate signal decays from background decays and to understand the impact of selection criteria on decays present in data. 
%Many different simulated decay types have been used over time for the development of the selection and analysis of \bsmumu decays, 
The simulated decays used for studies documented in this thesis are listed in Table~\ref{tab:MC_decays} along with the data taking conditions and simulation versions used to generated the decays.

There exist multiple versions of the simulation because it is updated as understanding of the detector increases and to incoporate differences in data taking condidtions, such as the trigger lines or center-of-mass energy, present in each year of data is collected. Similar simulation versions must be used to compare different types of simulated decays or data taking condidations so that differences are not masked by variations in the simulation of the decays. The simulated decays in Table~\ref{tab:MC_decays} listed under the studies they are used in. 

Simulated \bbbarmumux decays are used to understand the combinatorial background of \bsmumu decays, however producing a large enough sample of these decays to be useful is computational expensive and produces large output files to save generated decays. Therefore cuts are applied at the generation level for \bbbarmumux decays to reduce the size of the samples that are saved and to speed production. The cuts, listed in Table~\ref{tab:MC_decays}, are applied on the muon momenta, the reconstructed mass of the muon pair, the product of the momenta of the muons and the distance of closest approach of the two muon. %The generator level cuts save a factor of 5 of what needs to be saved, striping filtering also reduces it by a factor of 10! The information is in the LHCb-ANA-2013-032, the 2012 bbbarmumux sample corresponds to 7fb-1.
  
%The development of the selection and analysis of \bsmumu decays requires the use of simuluated decays, as described in Section X. The reconstucted \bsmumu candidates come for a range of different different processes, as already discussed, in order to seperate real \bsmumu decays from the background, large clean samples of simulated decays are used so that the differences between signal and background decays can be understood. Futhermore simulated samples are needed to understand 

 
\begin{table}[ht]
\begin{center}
\begin{tabular}{llll}
\hline
Decay 			& Data taking conditions 	& Simulation Version 	& Generated Events per magnet polarity \\ \hline 
\multicolumn{4}{c}{{\it Stripping selection studies selection}}  \\ \hline 
\bsmumu			& 2012	& sim06b  		& 1 M			 \\ 
\bdmumu			& 2012	& sim06b  		& 1 M\\ 
\bdkpi			& 2012	& sim06b  		& 500 k\\ 
\bujpsik		& 2012	& sim06b  		& 500 k\\ \hline 
\multicolumn{4}{c}{{\it Multivarite classifer training}}  \\ \hline
\bbbarmumux, p $>$ 3 \gevc, 4.7 $< M_{\mu^{+} \mu^{-} <$ 6.0 \gevcc, DOCA $<$ 0.4mm, 1 $<$ PtProd $<$ 16~\gevc
                        & 2012  & sim06b                & 4.0 M      \\
\bbbarmumux, p $>$ 3 \gevc, 4.7 $< M_{\mu^{+} \mu^{-} <$ 6.0 \gevcc, DOCA $<$ 0.4mm, PtProd $>$ 16~\gevc
                        & 2012  & sim06b                & 3.3 M\\
\bsmumu                 & 2012  & sim06b                & 1 M \\ \hline
\multicolumn{4}{c}{{\it Analysis method development}}  \\ \hline 
\bsmumu			& 2011 	& sim08a   		& 0.3 M		  \\ 
       			& 2012 	& sim08i  		& 1 M			 \\ 
        		& 2015	& sim09a  		& 1 M	 \\ 
        		& 2016	& sim09a  		& 1 M ? \\ %Is this correct? I thought in the ntuples we have a lot more 2015 than 2016 MC? 
\bdkpi			& 2011	& sim08b  		& 0.4 M  \\ %11102003
        		& 2012	& sim08g  		& 4.3M \\ 
        		& 2015	& sim09a  		& 2 M  \\ 
        		& 2016	& sim09a  	 	& 4.1 M \\ 
\bskk   		& 2012	& sim08g  		& 3.6 M \\ %13102002, 2016 is sim09a 4.1 M per pol, 2011 is sim-8b and 0.8 M per pol
        		& 2015	& sim09a   		& 2 M \\  \hline
\end{tabular}
\vspace{0.7cm}
\caption{Simulated decays used for developing the selection and the analysis method listed according to the studies the decays are used in. Requirements are imposed on \bbbarmumux decays as they decays are generated, the cuts are included alongside the decay type.}
\label{tab:MC_decays}
\end{center}
\end{table}

%Simulated \bsmumu, \bdmumu, \bdkpi and \bujpsik decays for 2012 data taking condition are used for studying the stripping selection in Section X.

%The training and testing of multivarite classifiers in Section X uses simulated \bsmumu and \bbbarmumux decays for for 2012 data taking conditions.
 
%Simulated events for \bsmumu, \bskk and \bdkpi for data taken in 2011, 2012, 2015 and 2016 are used for developing the analysis method in Chapter X. 

%The production of simulated events is constantly being developed as understanding of the detector increases and to include changes made for each data is recorded at theLHC. Therefore there exists a number of different simulation versions that can be used to simulate events.

%Each year data is collected at LHCb the conditions the experiment operates at and the proton collisions delivered by the LHC change. These changes include differences in the the selection used in the trigger for each year and increases in the centre of mass energy of proton collisions. 

%Therefore to understand data collected in different years simulated events from each year of data taking is needed, it is important to use similar simulation versions for each year so that the difference in the data taking conditions are not masked by differences in simulation versions. Simiarly for training multivarite classifiers consistent simulation versions are needed for the signal and background samples so that difference between signal and background distributions are not masked by differences in simulation versions.

%In general the stripping selections are applied to simulated events, however events that do not pass the stripping selection are still saved and can be used after reprocessing the simulated events. However simulation conditions can be set up so that events that do not pass the stripping selection are discarded and can never be used and also cuts can be applied on particles when they are generated before the detector response is simulated and the events are reconstructed. This is used when a very large same of simulated events needs to be generated in order to have a suitably large same of events reconstructed and is the case for the samples of \bbbarmumux simulated events.

%Two simulated samples of \bbbarmumux is used to understand the long lived combinatorial background and for the training of the multivariate classifier in Section \ref{sec:BDT}. For these samples events that did not pass the stripping selection have not been saved and requirements were applied to the generated events. 
%Simulated \bsmumu events with the same simulation version are also used for training the multivariate classifier to keep the simulation condition consistent.
On the whole simulated decays accurately model what occurs in data, however there are a couple of area where the simulation falls short of reality.
%Although in general simulated events accurately model what occurs in data there are several areas where this is not the case. 
The distributions of particle identification variables and properties of the underlying proton-proton collision, such as the number of tracks in an event, are not well modelled in simulation. %Have I said what an event is?
The mis-modelling of particle identification variables can be corrected for using the PIDCalib package and simulated decays can be re-weighted using information from data to accuratly model the under lying event, this re-weighting is described in Section X. 


\subsection{Trigger}
\label{sec:triggerRequirements}

The trigger is the first step in the selection process and the structure of the trigger is described in Section X. Since \bsmumu decays are very rare a broad set of trigger requirements is used in order to keep a high proportion of \bsmumu decay at this step of the selection. Specific trigger lines are not used in the selection but rather the combined results of a large selection of trigger lines at each level of the trigger. The combinations of trigger lines used are the L0Global, Hlt1Phys and Hlt2Phys triggers. The L0Global tigger combines all trigger lines present in the L0 trigger, it selects an event provided at least one L0 selects it and rejects an event if no L0 trigger selects it. The Hlt1Phys and Hlt2Phys triggers are very similar to the L0Global trigger except that decisions are based only trigger lines related to physics processes and HLT trigger lines used for calibration are excluded. 

Different trigger decisions on these lines are used to select decays for the Branching Fration and effective lifetime analyses. The Branching fraction selection imposed the loosest trigger requirements by requiring a event to pass the `Dec' decision at each trigger level as illustrated in set `A' of Table X. Trigger decisions are defined in Section X. The effective lifetime analysis has slightly more constrained trigger requirement, requiring an event passes either the `TIS' or `TOS' decision at each level of the trigger as illustrated in set `B' of Table X. The trigger choice for the effective lifetime is motivated by the determination of the acceptance fuction in Section X. 
%The selection criteria used in trigger lines and the specific lines included in the trigger change with each year of data taking, the dominant lines for triggering \bsmumu decays for each year are shown in Table~X. 

Events are required to be either TIS, triggered independent of signal or TOS, triggered on signal, on the trigger lines used at each level of the trigger.
%The selection criteria used in trigger lines and the specific lines included in the trigger change with each year of data taking, the dominant lines for triggering \bsmumu decays for each year are shown in Table~X. 
Slightly different trigger requirements are used to select \bhh decays used to develop and validate the effective lifetime analysis, the same broad trigger lines are used but the requirement on the output varies depending on the use of the \bhh events. The are two sets of trigger requirments, set `A' and `C', in Table~\ref{tab:triggers} are used to select \bhh decays, it will be made clear in later sections where \bhh decays are used which trigger requirements are imposed. 

\begin{table}[ht]
\begin{center}
\begin{tabular}{ll}
\hline
Trigger Line	& Trigger decision \\ \hline
\multicolumn{2}{c}{{\it set A}} \\ \hline
L0Global	& Dec\\
Hlt1Phys	& Dec \\
Hlt2Phys	& Dec \\ \hline
\multicolumn{2}{c}{{\it set B}} \\ \hline
L0Global	& TIS or TOS \\
Hlt1Phys	& TIS or TOS \\
Hlt2Phys	& TIS or TOS \\ \hline
\multicolumn{2}{c}{{\it set C}} \\ \hline
L0Global	& TIS\\
Hlt1Phys	& TIS \\
Hlt2Phys	& TIS \\ \hline
\end{tabular}
\vspace{0.7cm}
\caption{Trigger lines used to select \bsmumu and \bhh decays. Set `A' is used to select decays for the Branching Fraction analysis. Set `B' is used to select \bsmumu decays for the effective lifetime anaylsis. Sets `A' and `C' are used to select \bhh decays used to develop the \bsmumu effective lifetime analysis.}
\label{tab:triggers}
\end{center}
\end{table}

% An alternative table and therefore the text needs editing.

\begin{landscape}
\vspace*{\fill}
\begin{table}[ht]
%\begin{center}
\centering
\begin{tabular}{lllll}
\hline
	   &\multicolumn{4}{c}{Trigger decision} \\ \hline
\cline{2-5}
Trigger Line & Branching Fraction analysis & \bsmumu effective lifetime &\multicolumn{2}{c}{\bhh effective lifetime} \\
             &                             &                            & set A      & set B \\
\cline{4-5}
L0Global	& Dec                      & TIS or TOS                 & TIS        & Dec \\
Hlt1Phys	& Dec                       & TIS or TOS                 & TIS        & Dec \\
Hlt2Phys	& Dec                      & TIS or TOS                 & TIS        & Dec \\ \hline
\end{tabular}
\vspace{0.7cm}
\caption{Trigger lines used to select \bsmumu and \bhh decays. Set `A' is used to select decays for the Branching Fraction analysis. Set `B' is used to select \bsmumu decays for the effective lifetime anaylsis. Sets `A' and `C' are used to select \bhh decays used to develop the \bsmumu effective lifetime analysis.}
\label{tab:triggers2}
%\end{center}
\end{table}
\vspace*{\fill}
\end{landscape}
There was a problem with the implementation of the Hlt2Phys Dec decision in 2016 simulated events.%, the decision returned was always 1.  
This only affect the selection of \bhh decays. In order to emulate this trigger a combination of Hlt2 lines that select \bhh events, listed in Table~\ref{tab:HltDecEmulation}, is used instead of HLT2Phys when the Dec decision is required. 

\begin{table}[ht]
\begin{center}
\begin{tabular}{l}
\hline
\bhh trigger lines \\ \hline
Hlt2Topo2BodyDecision Dec  \\
Hlt2B2HH Lb2PPiDecision Dec \\
Hlt2B2HH Lb2PKDecision Dec \\
Hlt2B2HH B2PiPiDecision Dec \\
Hlt2B2HH B2PiKDecision Dec \\
Hlt2B2HH B2KKDecision Dec  \\
Hlt2B2HH B2HHDecision Dec \\ \hline

\end{tabular}
\vspace{0.7cm}
\caption{Trigger lines used to emulate the Hlt2Phys_Dec decision for \bhh data and simulated events.}
\label{tab:HltDecEmulation}
\end{center}
\end{table}

\subsection{Loose Selection}
\label{sec:stripping}
%http://lhcb-release-area.web.cern.ch/LHCb-release-area/DOC/stripping/config/stripping20/dimuon/strippingbs2mumulineswidemassline.html


The stripping selection, as described in Section~\ref{Software_Simulation}, is applied to all events that pass the trigger. It consists of individual stripping lines that select reconstructed candidates for specific decays by exploting differences between the decays and the backgrounds that mimic them. 

The selection of \bsmumu and \bhh decays for the \bsmumu effective lifetime measurement uses the same stripping lines as those in the \bmumu Branching Fraction measurements. 

%The stripping lines used in the last published Branching Fraction analysis 


These lines were designed at the start of Run~1 by studying the efficiencies of different selection cuts from simulated events \cite{}. However since then improvements have been made to the simulation of particle decays at LHCb, therefore it is prudent to check the accuracy of the selection efficiencies with updated simulated events and investigate where improvements can be made to the efficiency of the stripping selection used to select \bsmumu events.  %The stripping lines have been developed to imposes only loose requirements so that as much information as possible is still avaliable to develop the analysis and understand background events after the stripping selection.

Furthermore,  before the start of Run~2 during the long shut down the stripping selection was re-run on the data collect in Run~1 allowing changes to be made to existing stripping lines and new lines to be added. The stripping selection and efficiency studies are described in Sections~\ref{strippingold} and \ref{strippingstudies}. 

%In general the stripping selection imposes only loose requirements so that as much information as possible is still avaliable to develop the analysis and understand background events after the stripping selection. 

 
%REDO THIS AFTER WORKING OUT EXACTLY WHAT DETAILS I THINK ARE IMPORTANT TO INCLUDE!
%The stripping selection is a set of loose cuts that are applied to reconstructed events that have passed the trigger. The stripping selection consist of `lines' that are taylored to select particular decays. The aim of stripping lines is to reduce the size of the data sets collected by the experiment to a managable size on which tighter selection cuts to be developed and applied offline. Events that do not pass the selection cuts in the stripping lines are not directly avaliable to physics analyses. Therefore the cuts applied in the stripping lines are designed to remove obvious background events whilst keeping a high efficecny on the decay of interest. Restraints are placed on the amount of data that can pass the stripping selection for a particular analysis, typically this is set to be 0.05$\%$ of the original LHCb data set size for events that are saved in DST files. 
%This paragraph is ok.

%The events that pass the selection requirements in the trigger are reconstructed and the number of events is reduced by a set of loose selection cuts. These selection cuts are aimed at removing obvious background decays by exploiting the differences between real \bmumu decays and backgrounds that mimic them. The loose selection is composed to two parts; the stripping selection, outlined in Section~\ref{Software_Simulation}, and a set pre-selection cuts. 


%The Branching Fraction and effective lifetime analyses share the same loose selection, afterwards tighter cuts are applied to remove specific backgrounds many of which are tuned for each analysis, hence the loose selection cuts are called the pre-selection cuts. 

%The data set that comes out of the trigger is too large to be processed seperately for each physics analysis. The stripping selection, briefly introduced in Section~\ref{}, is a set of selection requirements taht are applied once and the output of the stripping is used for each physics analysis. The aim of the stripping selection is to reduce the size of the data set to a managable size that each analyst can use to develop their analysis. It is compased of many stripping `lines', each line consists fo a set of selection cuts that have been tuned to select particle decays relevant for each physics analysis and only events that pass a stripping line are avaliable to be used in physics analyses.  There are restrictions placed on the amount of data that can pass a stripping line, typically the output of a line must be less than 0.05~$\%$ of the original data set size. The analyses described in Chapters X and Y use the output of three stripping lines designed to select \bsmumu, \bhh and \bujpsik decays. 

%The stripping lines used in the last published Branching Fraction analysis were designed at the start of Run ~1 by studying the efficiencies of different selection cuts from simulated events \cite{}. However since then improvements have been made to the simulation of particle decays at LHCb, therefore it is prudent to check the accuracy of the selection efficiencies with updated simulated events and investegate where improvements can be made to the efficiency of the stripping selection used to select \bsmumu events.  Furthermore,  before the start of Run~2 during the long shut down the stripping selection was re-run on the data collect in Run~1 allowing changes to be made to existing stripping lines and new lines to be added. The stripping selection and efficiency studies are described in Sections~\ref{strippingold} and \ref{strippingstudies}. %The changes discussed in Section X were included in this re-running of the stripping selection and are inlucded in the stripping lines used for Run~2 data.
%In general the stripping selection imposes only loose requirements so that as much information as possible is still avaliable to develop the analysis and understand background events after the stripping selection. Therefore after the stripping selection further loose selection requirements are applied to remove different background decays before the tighter selection requirements detailed in Section~\ref{finalloosesel}.



%The following sections describe the stripping selection used in previous analysis, a study to improve \bsmumu selection efficicencies of the stripping and a summary the the final choice of cuts in the stripping lines and other loose preselection cuts that are applied. 

\subsubsection{Run 1 Stripping Selection}
\label{strippingold}
The stripping selection cuts and cuts applied during the reconstruction of particle decays for the Run 1 \bmumu Branching Fraction analysis \cite{} to select \bmumu, \bhh and \bujpsik are shown in Table X. The selection of \bujpsik and \bhh decays is kept as similar as possible to the selection of \bsmumu decays to avoid introducing systematic errors when \bhh and \bujpsik decays are used in the normalisation for the Branching Fraction measurement. The selection of \bujpsik event must diverge from the \bsmumu selection due to additoinal particles in the final state of the decay. The stripping selection imposes more cuts to select \bhh decays compared to \bsmumu because \bhh decays are much more abundant therefore extra cuts are needed to reduce the number of events passing the stripping to an acceptable level. The cuts applied to \bhh in the stripping are the later applied to \bsmumu events after the stripping selection. 


\begin{landscape}
\vspace*{\fill}
\begin{table}[ht]
\begin{center}
\begin{tabular}{l|lll}
\hline
  Particle              & \bsmumu                                     & \bhh                            &\bujpsik       \\
\hline             
\bs or $B^{+}$         & |M - M$_{PDG}$| $<$ 1200 \mevcc              & |M - M$_{PDG}$| $<$ 500 \mevcc    & |M - M$_{PDG}$| $<$   500 \mevcc   \\          
                      & DIRA > 0                                    & DIRA > 0                          & Vertex $\chi^{2}$/ndof < 45    \\       
                      & FD $\chi^{2}$ $>$ 121                        & FD $\chi^{2}$ $>$ 121             & IP $\chi^{2}$ $<$ 25  \\ 
                      & IP $\chi^{2}$ $<$ 25                         & IP $\chi^{2}$ $<$ 25              &         \\            
                      & Vertex $\chi^{2}$/ndof < 9                   & Vertex $\chi^{2}$/ndof < 9        &         \\   
                      & DOCA $<$ 0.3 mm                             & DOCA $<$ 0.3 mm                   &         \\               
                      &                                             & $\tau$ $<$ 13.248 \ps             &         \\
                      &                                             & $p_{T}$ $>$ 500 \mevc             &          \\
\hline   
\jpsi                  &                                             &                                   & |M - M$_{PDG}$| $<$   60 \mevcc   \\
                      &                                             &                                   & DIRA > 0    \\
                      &                                             &                                   & FD $\chi^{2}$ $>$ 169 (225)  \\
                      &                                             &                                   & Vertex $\chi^{2}$/ndof < 9        \\  
                      &                                             &                                   &   DOCA $<$ 0.3 mm       \\  
\hline             
Daugther $\mu$ or $h$   & Track $\chi^{2}$/ndof < 3                 & Track $\chi^{2}$/ndof < 3           & Track $\chi^{2}$/ndof < 3     \\       
                        & isMuon = True                             &                                    & isMuon = True           \\ 
                        & Minimum IP $\chi^{2}$ $>$ 9               & Minimum IP $\chi^{2}$ $>$ 9         & Minimum IP $\chi^{2}$ $>$ 25     \\                   
                        &                                           & 0.25 \gevc $<$ $p_{T}$ $<$ 40 \gevc &  \\
                        &                                           & $p$ < 350 \gevc                     &  \\
                        &                                           & ghost probability $<$ 0.3 (0.4)     &  \\
\hline
$K^{+}$                 &                                           &                                     & Track $\chi^{2}$/ndof < 3   \\
                       &                                           &                                     & $p_{T}$ $>$ 0.25 \gevc  \\
                       &                                           &                                     & Minimum IP $\chi^{2}$ $>$ 25 \\
%                       &                                           &                                     &  isLong = True  \\ 
\hline
\end{tabular}
\vspace{0.7cm}
\caption{Selection requirements applied during the stripping selection for Run~1 data used in the \bmumu Branching Fraciton analysis \cite{} to select \bmumu, \bhh and \bujpsik decays. The track $\chi^{2}$/ndof and isMuon cut are applied during the reconstuction.}
\label{tab:PreviousStripping}
\end{center}
\end{table}
\vspace*{\fill}
\end{landscape}

The variables used in the stripping selection are:
\begin{itemize}
\item the reconstructed mass, $M$, - the mass and momenta of the decay products of the $B$ meson (or \jpsi) are combined to provide its reconstructed mass. Cuts on the mass remove events with a reconstructed mass far from the expected particle mass and are therefore clearly to be background events. Loose mass requirements are made on for the \bsmumu selection to allow for the study of semi-leptonic backgrounds that have a mass less than the \bs mass when mis-identified as \bsmumu decays;
\item the ``direction cosine'', DIRA, - this is the cosine of the angle between the mometum vector of the $B$ meson and the vector connecting the primary vertex to the secondary vertex where the $B$ meson decays. For correctly reconstructed events the direction cosine should be very close to one, requiring events to have positive value ensuring events are travelling in the incorrect direction form the SV to the PV are removed;
\item the flight distance (FD) \chisqd - this is computed by performing the fit for the production vertex of a particle but including the tracks from its decay products that originate from the decay vertex in the fit as well. For a $B$ meson the FD \chisqd is likely to be large because $B$ mesons have long lifetimes therefore the tracks of its decays products will not point back to the production vertex. Alternatively a \jpsi will have a small \chisqd because it decays instantaneously;
\item track fit \chisqd/$ndof$ - provides a measure of the quality of a fitted track, placing an upper limit on this parameter removes poor quality tracks and therefore backgrounds composed of poorly reconstructed decays;
\item vertex fit \chisqd/$ndof$ - provides a measure of how well tracks can be combined to form a vertex, placing an upper limit on this parameter removes poorly constrained verticies and therefore backgrounds composed of poorly reconstructed decays;
\item ``distance of closest approach'' (DOCA) - this is the distance of closest appoach of two partilcles computed from the straight tracks in the VELO. For the decay products of a particle, for example the muons from \bsmumu, this distance would ideally be zero because they originate from the same vertex;
\item decay time \lt - this is the length of time a particle lives for as it travels from its production vertex to its decay vertex. Applying an upper decay time cut removes unphysical background decays;
\item isMuon - particle identification variable defined in Section \ref{} that returns True for muons and False for other particles;
\item transverse mometum, \pt - the component of a particle's momentum perpendicular to the beam axis. Decay products of $B$ mesons are expected to have relatively high \pt values due to the heavy $B$ meson masses however an upper limit removes unphysical backgrounds
\item momentum, \p - an upper limit on the mometum of a particle  removes unphysical backgrounds
\item ghost probability - defined in Section \ref{} provides the probability of a tracking being composed on random hits in the detector.
\item impact parameter (IP) \chisqd - this is the change in the fit for a primary vertex caused by removing one track in the fit. In a \bsmumu decay, the \bs is produced at the PV therefore it should have a small IP \chisqd value whereas the muons will be displace from the PV because of the realtively long lifetime of the \bs and therefore will have a large IP \chisqd;
\item Minimum muon impact parameter (IP) \chisqd - this is the IP \chisqd of the muons with respect to all PVs in the event, this is to remove prompt muons created at any PV in the event and therefore reduce the prompt combinatorial background. 
\end{itemize}



\subsubsection{Improving Stripping Selection Efficiency for \bmumu decays}
\label{strippingstudies}

\begin{landscape}
\vspace*{\fill}
\begin{table}[ht]
\begin{center}
\begin{tabular}{p{6cm}lll}
                  & \multicolumn{3}{c}{Efficiency}  \\ 
\cline{2-4}
Requirement                                  & \bsmumu                   & \bhh                &\bujpsik  \\
\hline
$B$ |M - M$_{PDG}$|                           & (100.00 $\pm$ 0.00)$\%$  & ( $\pm$ )$\%$        & ( $\pm$ )$\%$ \\
\bsd or \jpsi DIRA                            & (99.43 $\pm$ 0.01) $\%$  & ( $\pm$ )$\%$        & ( $\pm$ )$\%$ \\
\bsd or \jpsi FD $\chi^{2}$                   & (83.89 $\pm$ 0.06) $\%$  & ( $\pm$ )$\%$        & ( $\pm$ )$\%$ \\
\bsd or \jpsi IP $\chi^{2}$                   & (96.88 $\pm$ 0.03) $\%$  & ( $\pm$ )$\%$        & ( $\pm$ )$\%$ \\
\bsd or \jpsi vertex $\chi^{2}$/ndof          & (97.36 $\pm$ 0.03) $\%$  & ( $\pm$ )$\%$        & ( $\pm$ )$\%$ \\
\bsd or \jpsi DOCA                           & (99.86 $/pm$ 0.01) $\%$   & ( $\pm$ )$\%$        & ( $\pm$ )$\%$ \\               
\hline
%$\mu$ or $h$ track $\chi^{2}$/ndof          & -                         & ( $\pm$ )$\%$        & ( $\pm$ )$\%$ \\
%$\mu$ or $h$ isMuon                         & -                         & ( $\pm$ )$\%$        & ( $\pm$ )$\%$ \\
$\mu$ or $h$ minimum IP $\chi^{2}$           & (80.47 $\pm$ 0.06) $\%$  & ( $\pm$ )$\%$        & ( $\pm$ )$\%$ \\
\hline
Efficiency after above cuts                  & (73.75 $\pm$  0.07) $\%$  & ( $\pm$ )$\%$        & ( $\pm$ )$\%$ \\
\hline
Efficiency after all striping line cuts      & (73.75 $\pm$  0.07) $\%$  & ( $\pm$ )$\%$        & ( $\pm$ )$\%$ \\

\end{tabular}
\vspace{0.7cm}
\caption{Stripping line efficiencies for \bsmumu, \bhh and \bujpsik  2012 simulated decays after broad trigger requirements. Selection cuts applied are listed in Table~\ref{}.  }
\label{tab:Run1strippingEff}
\end{center}
\end{table}
\vspace*{\fill}
\end{landscape}

The efficiencies of the stripping lines for selecting \bmumu, \bhh and \bujpsik decays are shown in Table \ref{tab:Run1strippingEff}. The efficiencies are evalutated using 2012 sim06 simulated events and a loose set of trigger requirements (set A from Table~\ref{tab:triggers}) have been imposed. Only cuts that are applied to select \bmumu decays are evaluated, the efficiencies for the isMuon and track $\chi^{2}$/$ndof$ are not included because decays that do not pass these requirements are not included in the samples of simulated events. 

The selection efficiencies are similar accross the different decays for the selection cuts that are shared with the \bmumu selection, and the total efficicency is around 70~$\%$ for all decays. 

The similarity of selection efficiencies across the different decays is futher illustrated in Figure~\ref{fig:ratio_plots} which shows the ratio of the efficiencies $B^{0}_{(s)}\to\mu^{+} \mu^{-}$ and $B^{+}\to J/\psi K^{+}$ where each cut has been applied independantly.  With the exception of the IP $\chi^{2}$ cuts on the daughter particles, the ratio of efficiencies is well within $2\%$ of 1 for the range of cuts values shown. The ratio of the \bsmumu and \bujpsik efficiencies for the daughter particle IP $\chi^{2}$ markedly deviates from unity, showing that the IP $\chi^{2}$ distribution of the muons and kaon are very different as seen previous in \cite{Diego}. If the other selection cuts are applied to the simularted events before the daughter IP $\chi^{2}$ requirement the ratio of \bmumu and \bujpsik efficiencies is much closer to 1. 


\begin{figure}
    \centering
    \begin{subfigure}[b]{0.4\textwidth}
        \includegraphics[width=\textwidth]{./Figs/Selection/IPS.png}
        \caption{ }
        \label{fig:IPS_ratio}
    \end{subfigure}
    ~ %add desired spacing between images, e. g. ~, \quad, \qquad, \hfill etc. 
      %(or a blank line to force the subfigure onto a new line)
    \begin{subfigure}[b]{0.4\textwidth}
        \includegraphics[width=\textwidth]{./Figs/Selection/CHI2.png}
        \caption{ }
        \label{fig:CHI2_ratio}
    \end{subfigure}
    ~ %add desired spacing between images, e. g. ~, \quad, \qquad, \hfill etc. 
    %(or a blank line to force the subfigure onto a new line)

    \begin{subfigure}[b]{0.4\textwidth}
        \includegraphics[width=\textwidth]{./Figs/Selection/DOFS.png}
        \caption{ }
        \label{fig:FD_ratio}
    \end{subfigure}
   \begin{subfigure}[b]{0.4\textwidth}
        \includegraphics[width=\textwidth]{./Figs/Selection/daug_IPS.png}
        \caption{ }
        \label{fig:IPS_ratio}
    \end{subfigure}
    \caption{The ratio of $B^{0}_{(s)}\to\mu^{+} \mu^{-}$ to $B^{+}\to J/\psi K^{+}$ stripping efficiencies on MC events when each cut has been applied independently of all other cuts. The current cut values are marked by the blue lines.}
    \label{fig:ratio_plots}
\end{figure}


The efficiencies for most of the stripping cuts is \sim 97~$\%$, however, the efficienices of the cuts on the FD $\chi^{2}$ of the \bsd or \jpsi and the daugher IP $\chi^{2}$ of the muon or hadron pair are lower at $83\%$ and $80\%$, respectively. Therefore improvements to the stripping selection efficiencies could be acheived by altering these two selection requirements. 


The set of events removed by each cut in a stripping selection is not independant. Therefore the effect of changing on cut on the total efficiencyu of a stripping selection must be considered. Figure~\ref{fig:efficiencyplots} shows the total efficiency of the \bsmumu stripping line on simulated \bsmumu events that have passes the trigger requirement for a range of cut values for the FD $\chi^{2}$ and daugher IP $\chi^{2}$ requirements. As expected the lower the cut values are the more efficient the stripping line becomes. However one of the main purposes of the stripping selection is to reduce the size of the data set by removing obvious background events, therefore the cuts cannot be set as loose as possible. The curve on Figure~\ref{fig:efficiencyplots} is used as a guide to study a set of FD $\chi^{2}$ and daughter IP $\chi^{2}$ cut values in more depth, taking in to account both the signal efficiency and the amount of data retained by the selection, the set of chosen cuts aims to keep both cuts as tight as possible for a certain efficiency. Any changed applied to the \bmumu stripping line must be propagated through into the stripping lines for \bhh and \bujpsik decays in order to keep the selection as similar as possible for across all the decays. Changes to the \bsd FD $\chi^{2}$ in \bsmumu correspond to a change in the \bsd FD $\chi^{2}$ in \bhh and the \jpsi FD $\chi^{2}$ for \bujpsik, similarly changes to the muon IP $\chi^{2}$ in \bsmumu correspond to changes in the hadron IP $\chi^{2}$ in \bhh and the muon IP $\chi^{2}$ in \bujpsik. Table~\ref{tab:eff_and_retention} showns the total efficiencies of the \bmumu, \bhh and \bujpsik stripping lines along side the amount of data retained for the set of cuts illustrated in Figure~\ref{fig:eff_contours}. The data retention is computed by applying the stripping selection to a sub-set of 2012 data, with the trigger requirements applied, to find the number of events that pass the stripping lines for each pair of FD $\chi^{2}$ and daughter IP $\chi^{2}$ cuts. The number of events for each set of cuts is normalised to the number of events passing the original Run~1 stripping line requirements in order to show the fractional increase caused by loosening the cut values. 


\begin{figure}
    \centering
    \begin{subfigure}[b]{0.4\textwidth}
        \includegraphics[width=\textwidth]{./Figs/Selection/strip_chart.png}
        \caption{ }
        \label{fig:eff}
    \end{subfigure}
    ~ %add desired spacing between images, e. g. ~, \quad, \qquad, \hfill etc. 
      %(or a blank line to force the subfigure onto a new line)
    \begin{subfigure}[b]{0.4\textwidth}
        \includegraphics[width=\textwidth]{./Figs/Selection/strip_chart1.png}
        \caption{ }
        \label{fig:eff_contours}
    \end{subfigure}
    \caption{Efficiency figures.}
    \label{fig:efficiencyplots}
\end{figure}


(Not explained that some cuts are in the reconstruction very well, so are applied before the stripping and I cannot change them and I've not made the trigger requirements clear either or probably exactly what cuts are applied to get the efficiencies, I've not included the B2JpsiK plots - do I want to? I don't think they add much to the dicussion, they show a comparison of efficiencies and that it's flat - I can ask Harry if it would add much to the discussion).
\begin{landscape}
\vspace*{\fill}
\begin{table}[htbp]
\begin{center}
\begin{tabular}{ ll|lll|lll}\hline
                       &                     & \multicolumn{3}{c}{Stripping line efficiency} &  \multicolumn{3}{c}{Stipping line retention} \\
\cline{3-8}
\bsd, \jpsi FD \chisqd & \mu, $h$ IP \chisqd & \bsmumu & \bdkpi   & \bujpsik & \bsmumu & \bdkpi & \bujpsik \\
\hline
15                     &5.00                 & XX $\%$ &  XX $\%$ &  XX $\%$ &  XX      & XX     & XX        \\
14                    &4.25                  & XX $\%$ &  XX $\%$ &  XX $\%$ &  XX      & XX     & XX        \\
13                    &4.00                  & XX $\%$ &  XX $\%$ &  XX $\%$ &  XX      & XX     & XX        \\
12                    &3.50                  & XX $\%$ &  XX $\%$ &  XX $\%$ &  XX      & XX     & XX        \\
11                    &3.00                  & XX $\%$ &  XX $\%$ &  XX $\%$ &  XX      & XX     & XX        \\
10                    &2.75                  & XX $\%$ &  XX $\%$ &  XX $\%$ &  XX      & XX     & XX        \\
9                     &2.50                  & XX $\%$ &  XX $\%$ &  XX $\%$ &  XX      & XX     & XX        \\
\hline
\end{tabular}
\end{center}
\label{tab:eff_and_retention}
\vspace{0.7cm}
\caption{Retention of data and stripping line efficiencies. Efficiencies are \bdkpi but retention is \bhh because the stripping line selects all \bhh decays, $h$ is $K$ or $\pi$. }
\end{table}
\vspace*{\fill}
\end{landscape}

An increase of $16\%$ can be gained in the stripping selection efficiencies by using the loosest cuts in Table \ref{tab:eff_and_retention} however the loosest cuts increase the amount of data passing the \bmumu stripping selection by a factor of 7. The final set of cuts used in the stripping selection must be a compromise between the selection efficiency and the amount of data that passes the selection. The selection cuts of \bs FD $\chi^{2}$ $>$ 121 and minumum muon IP $\chi^{2}$ $>$ 9 would increase the \bmumu selection efficiency by from 71~$\%$ to 82~$\%$ and the amount of data retained would be doubles. The increase of the data retained by the \bhh and \bujpsik lines is smaller and the efficiencies are similar to the \bmumu selection efficiencies. Therefore these cuts are applied in the stripping selection for this analysis. However the increase in efficiency after the stripping selection will not necessarily be propagated through the whole analysis. 



\subsection{Final Selection and Pre-selection}
\label{finalloosesel}
The stripping selection requirements along with a number of other loose selection cuts to select \bmumu, \bhh and \bujpsik decays in 2011, 2012, 2015 and 2016 are shown in Table~\ref{tab:fullpreselection}. The selection includes the looser stripping cuts on the \bsd and \jpsi FD \chisqd and muon and hadron IP \chisqd. Additionally the selection of \bmumu decays includes the momentum, ghost track probabilty and decay time cuts made in the \bhh stripping line, but were absent in the \bmumu stripping line. Several cuts are included to remove specific background events, the mass requirement on \bmumu decays is tightened to ensure $B_{S}^{0}\to\mu^{+}\mu^{-}\gamma$ decays are not within the mass window and a lower bound is placed on the $B$ meson transverse mometum to remove pairs of muons originating from $pp \to p\mu\mu p$ decays. %Something about that decays with mass above 6000 MeV are used for background studies?
The decays that pass the selection cuts in Tables~\ref{tab:fullpreselection} in both data and simulated events are used to develop tigher selection requirements. 


\begin{landscape}
\vspace*{\fill}
\begin{table}[ht]
\begin{center}
\begin{tabular}{llll}
Particle                & \bsmumu                                     & \bhh                                 & \bujpsik \\
\bs or $B^{+}$          & 4900 \mevcc $<$ M $<$ 6000 \mevcc           &  |M - M$_{PDG}$| $<$   500 \mevcc      & |M - M$_{PDG}$| $<$   500 \mevcc   \\                              
                        & DIRA > 0                                    & DIRA > 0                             & Vertex $\chi^{2}$/ndof < 45    \\       
                        & FD $\chi^{2}$ $>$ 121                       & FD $\chi^{2}$ $>$ 121                 & IP $\chi^{2}$ $<$ 25  \\               
                        & IP $\chi^{2}$ $<$ 25                        & IP $\chi^{2}$ $<$ 25                  &   \\             
                        & Vertex $\chi^{2}$/ndof < 9                  & Vertex $\chi^{2}$/ndof < 9            &   \\                  
                        & DOCA $<$ 0.3 mm                             & DOCA $<$ 0.3 mm                      &    \\                    
                        & $\tau$ $<$ 13.248 \ps                       & $\tau$ $<$ 13.248 \ps                &    \\
                        & $p_{T}$ $>$ 500 \mevc                        & $p_{T}$ $>$ 500 \mevc                &    \\
\hline    
\jpsi                   &                                             &                                      & |M - M$_{PDG}$| $<$   60 \mevcc   \\
                        &                                             &                                      & DIRA > 0    \\
                        &                                             &                                      & FD $\chi^{2}$ $>$ 121  \\
                        &                                             &                                      & Vertex $\chi^{2}$/ndof < 9        \\  
                        &                                             &                                      &   DOCA $<$ 0.3 mm       \\  

\hline
Daugther $\mu$ or $h$   & Track $\chi^{2}$/ndof < 3 (4)               & Track $\chi^{2}$/ndof < 3 (4)         & Track $\chi^{2}$/ndof < 3     \\                                           
                        & isMuon = True                               &                                      & isMuon = True           \\        
                        & Minimum IP $\chi^{2}$ $>$ 9                 & Minimum IP $\chi^{2}$ $>$ 9           & Minimum IP $\chi^{2}$ $>$ 25     \\                       
                        & 0.25 \gevc $<$ $p_{T}$ $<$ 40 \gevc         & 0.25 \gevc $<$ $p_{T}$ $<$ 40 \gevc   &
                        & $p$ < 500 \gevc                             & $p$ < 500 \gevc                      &
                        & ghost probability $<$ 0.3 (0.4)             & ghost probability $<$ 0.3 (0.4)      &
\hline
$K^{+}$                 &                                             &                                      & Track $\chi^{2}$/ndof < 3   \\
                       &                                             &                                      & $p_{T}$ $>$ 0.25 \gevc  \\
                       &                                             &                                      & Minimum IP $\chi^{2}$ $>$ 25 \\
                       &                                             &                                      &  isLong = True  \\ 
\hline
%
\hline
\end{tabular}
\vspace{0.7cm}
\caption{Loose selection cuts applied to select \bsmumu, \bhh and \bujpsik decays, where selection is different between Run~1 and Run~2 the Run~2 values are shown in parenthesis next to the Run~1 values.}
\label{tab:fullpreselection}
\end{center}
\end{table}
\vspace*{\fill}
\end{landscape}


\subsection{Pre-selection/Offline selection?}
\label{sec:offline_sel}
The output of the stripping selection still includes many background decays, further cuts shown in Table~\ref{} reduce the background decays. Some selection cuts are designed to remove specific background decays and the selection for \bsmumu decays used in the Branching Fraciton and effective lifetime analyses starts to diverge slightly.

The BDTS is a multivariate classifer that is designed to reduce the number of combinatorial background events. It is a Boosted Decision Tree (BDT) (see Section~\ref{} for a detailed description) that is trained on \bsmumu and \bbbarmumux simulated decays that have passed the \bmumu selection requirements in Table~\ref{}. The BDTS uses variables similar to those in the stripping selection to classify events;
\begin{itemize}
\item impace parameter \chisqd of the \bds
\item vertex \chisqd of the \bsd
\item direction cosine of 
\item distance of closest apporach of the muons
\item minimum impact paramter \chisqd of the muons with respect to all primary verticies in the event
\item impact paramter of the \bsd, this is the distance of closest approch of the $B$ to the primary vertex
\end{itemize}
The BDTS is applied to all candidates passing the \bmumu, \bhh and \bujpsik stripping lines, and candidates are required to have a BDTS value above 0.05. The chosen cut value has a efficiency of X $\%$ on \bsmumu decays and reject X $\%$ of \bbbarmumux decays. 


The semi-leptonic \bcjpsimunu decays when \jpsimumu contribute to the background of \bmumu decays when a muon from the \jpsi forms a good vertex with the muon from the $B_{c}^{+}$ decay. Due to the high mass of the $B_{c}^{+}$ this could place mis-reconstructed candidates within the \bs mass window. A `jpsi veto' can be used to remove background events from \bcjpsimunu decays. The veto works by removing events where one muon from the \bmumu candidate combined with any other opositely charged muon in the event has $\m_{\mu\mu} - m_{\jpsi}| < 30$~\mevcc. The veto has a rejection power of X  $\%$ on \bcjpsimunu events that have passed \bmumu selection cuts in Table~\ref{} and rejects only  $\%$ of \bmumu signal events. The expected number of \bcjpsimunu events after the full selection can be found in Section~X. 


The simiulated \bbbarmumux decays are used to train the multivarite classifier (Sect.~\ref{}) used in the Branching Fraction and effective lifetime analyses and variables included within the classifier (Sect.~\ref{}) as well as the BDTS. The simulated \bbbarmumux decays had tighter cuts on the FD \chisqd of the \bsd and minimum IP \chisqd of the muons than those listed Table~\ref{} applied when it was created. Events that did not pass those cuts are not avaliable for use in the classifier training, therefore these must be applied to all data and simulated decays to ensure the most effective and reliable performance of the multivarite classifers used in the analyses. 

The effective lifetime analysis requires that \bsmumu candidates have a dimuon invariant mass greater than 5320 \mevcc. The motivation comes from mass fit studies that are detailed in Section X. The consequence of this cut is to remove \bdmumu events and most background from mis-identified semi-leptonic and \bhh decays. The expect number of \bdmumu and mis-identified decays after the full selection can be found in Section~X.
\begin{landscape}
\vspace*{\fill}
\begin{table}[ht]
\begin{center}
\begin{tabular}{l|l|l}
\bsmumu                                     & \bhh                                 & \bujpsik \\
\hline
BDTS $>$ 0.05                              & BDTS $>$ 0.05                         & BDTS $>$ 0.05  \\       
\jpsi veto $\m_{\mu\mu} - m_{\jpsi}| < 30$~\mevcc &\jpsi veto $\m_{\mu\mu} - m_{\jpsi}| < 30$~\mevcc & - \\
\bsd FD \chisqd $>$ 225                     & \bsd FD \chisqd $>$ 225              & \jpsi FD \chisqd $>$ 225 \\
$\mu$ minimum IP \chisqd > 25              &$h$ minimum IP \chisqd > 25             &$\mu$ minimum IP \chisqd > 25     \\
M $>$ 5320 \mevcc                           &  -                                     & -  \\
\hline
\end{tabular}
\vspace{0.7cm}
\caption{Selection cuts applied to select \bsmumu, \bhh and \bujpsik decays.}
\label{tab:selection}
\end{center}
\end{table}
\vspace*{\fill}
\end{landscape}



\subsection{Particle Identification}
\label{sec:PID}
%Not a fan of this section :(
Particle identification (PID) variables are used to refine the selection of \bmumu candidates and to seperate different \bhh decays. 

In the selection of \bmumu decays PID variables are particularly useful to reduce the backgrounds coming from mis-identified semi-leptonic decays and \bhh decays and also help to reduce the number of combinatorial backgroudn events. 

The PID requirements to select \bmumu decays in the Branching Fraction and \bsmumu in the lifetime analysis are in Table~\ref{} alongside requirements to seperate \bhh decays. Two types of PID variables, defined in Section~\ref{PID_variables}, are used; DLL variables and ProbNN variables. DLL variables are useful to seperate \bhh decays where $h$ is either a pion or kaon because the varaibles compare different particle hypotheses with the pion hypotheses.

Requirements on ProbNN variables vary with the year of data taking because the classifers used in ProbNN variables are tuned to give the best performance depending on the different data taking conditions in the detector for each year. %The cuts are chosen to give similar efficiencies for each data sets at selecting signal and removing background accross the different years. 

Tighter PID requirements are used to select \bmumu decays for the Branching Fraction measurements compared to those used for the effective lifetime measurement. This is because more mis-identidied decays extend in to the \bd mass window than the \bs mass window therefore tigher requirements are necessary to reduce them to an acceptable level. %However this also reduces the efficiency for selecting \bmumu decays. 

{\it Prehaps a better way for the PID is to put the requirements in the text because the is could be clearer or maybe have two tables one for bsmumu and one for bhh? Also another consideration prehaps it would be clearere to seperate the effective lifetime and the BF after the stripping selection. I could clearly explain that the stripping and loose selection is used for both and therefore the BF decays must be considered in the stripping selection but after that the selection will focus on the effective lifetime. }


%\subsection{Multivariate Classifier}
%\label{sec:BDT}
%After the selection described so far the data set still contains predonominately background decayse mostly from long lived combinatorial background from \bbbarmumux decays. The most effective way to remove the remaining background decays without comprimising on the efficiency to selection \bsmumu decays is to apply a cut on the output of a multivariate classifier.

%A multivariate classfier is an algorithm that leanrs differences between signa and background events in the following way; the classifier is gien tow input samples, one contain only signal decays and the other containing just background decays and a set of input variables. THese input variables have different distributions for signal and background events. The classifier uses the distributions of ths input variables with it's knowledge of which events are signal and background to learn the difference between the two types of events. The algorithm is then be applied to a data set containing an unknown mixture of signal and background events and distinguish between them.  For each event the algotithm produces a number typically between -1 and +1 with high numbers indicating signal-like events and low numbers indicating background-like events. A cut can then be places on the output of the classifier to remove background events with a classifer response less thatn a particular value and the remaining daata set has a high purity for signal events. 

%To seperate the signal \bsmumu decays from the background decays a type of classifer called a Boost Decision Tree (BDT) is used. Other types of classifers were investigated however their performance was not as good. A BDT is made up of the combined outputs of seperate decision trees. A decision tree begins with a sample of event, where each event is know to be signal or background and a set of variables descrubing these events. The decision tree applied a cut on a variable that will be the most effective at seperating the signal and background events in the sample and creates two sub-samples. Another cut is then applied on each of the sub-samples to futher seperate signal from background. This process is repeated until a certain number of cuts, defined as the depth of the tree, or the number of events in each sub-sample has reached a minumum number. Each sub-sample produced at the end of the tree is called a leaf. The tree then uses it's knowledge of whehter an event is signal or background to assign each event the value of +1 or -1. An event is given a value +1 if it is in a leaf that is made up of a majority of signal eventsand an event is given the value -1 if it is in a leaf that has a majority of background events. The final decisions made by the tree are not prefect, some signal (background) events will be mis-classified as background events and given the value of -1 (+1). The process of decision making is illustrated in Figure~\ref{}. 

%One decision tree on its own is often not particulary good at classifying events, there is not way to correct mis-classified events in the leaves, and it is particularly sensitive to statistical fluctuations in the training samples. A BDT combined the output of numerous decision trees to improve the classification of events and reduce the depandance of the final decisions on statistical flucuations. A BDT starts with one decision tree and it then assigns weights to events in the sample depending on whether the output of the decision tree classified the events correctly or incorrectly. The weighted sample is then used as the input for the training of the next decision tree. The weights are designed so that the next tree is more likely to correly classify previously mis-classified events. This process is repeated until a certain number of trees have been trained. The reweighting process is know as boosting and the weighted applied to the samples are taken into account when combining the output of each decision tree into the overall output of the BDT. 

%The preformance a BDT can acheive depends on three main factors;
%\begin{itemize}
%\item the size of the training samples avaliable, the larger the training sample the more information the BDT can use to correctly classify events
%\item the input variables used to makes each decision in the trees - variables that have distinctly different distributions for signal and background events enable the classifier to split samples into sub samples with a high signal or background purity. (Also something about correlations?)
%\item the parameters that guide the training of the trees, such as the depth or miniumum number of events allowed in the leaves and the speed at which boosting occurs and also the total number of trees.
%\end{itemize}


%The preformance of a classifer depends on these three inputs in different ways but together they must be used to avoid overtraining of the algorthm. Thsi occurs when the final algorithm is extremely sensitive to statistical flucutations in the training samples and therefore preforms poorly when used on a statistically indpenedant set of events. Overtraining can be avoided by using a very large training sample and by limiting the number of trees in the BDT or the depth of each tree. Overtraining can be checked for by testing and training the algorithm on seperate samples. 


%(I think that before re-writing this section I should read the TMVA guide again, all the BDT parts both on boosting and also about what a BDT is it will given me good information to include and it also explains varibles such as nCuts. Why didn't I read that earlier?!? It also explains how a BDT is in sensitive to poorly discriminating variables and also how variables are ranked from the output of the BDT which will be useful when I explain how I chose the input variables. Prehaps once I have this chapter I should work out what key points I want to say where and they write again! But sometimes it is only through writing you can work this out. But then prehaps it should be; 1. short bullet points, 2. detailed bullet points and checking the order, 3. write!)

%The TMVA package was used to build several BDTs with different boosting techniques. The training and testing of the BDTs using adaptive boosting, gradient boosting and a novel uBoost methog is discussed in the following sections. The uBoost method is a novel method of boosting that creates an output with a uniform efficiency for a specific input variable that is not used in the training. This is particularly interesint for the measurement of the effective lifetime. The BDT output could be required to be uniform in the decay time which would simplify understanding the efficiency as it vaired with the decay time. This is discussed futher in another section. Other boosting methods were considered but didnot preform as well as the adaptive or gradient boosting or offer the novel method of the uBoost algorithm.

