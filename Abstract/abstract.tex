% ************************** Thesis Abstract *****************************
% Use `abstract' as an option in the document class to print only the titlepage and the abstract.
%\begin{abstract}
%This is where you write your abstract ...
%\end{abstract}
\chapter{Abstract}

This thesis documents studies of very rare B meson decays at the LHCb experiment on data taken during Run 1 and Run 2 of the Large Hadron Collider (LHC).


The LHCb experiment was designed to study the Standard Model of particle physics and new physics theories that go beyond the scope of the Standard Model through the decay of $b$-hadrons produced by high energy proton-proton collisions at the LHC. The measurements detailed in this thesis uses 1.0, 2.0 and 1.4 fb$^{-1}$ of data collected by the LHCb experiment in proton-proton collisions at center-of-mass energies of 7, 8 and 13 T$e$V.

The very rare decays $B_{s}^{0} \to \mu^{+} \mu^{-}$ and $B^{0} \to \mu^{+} \mu^{-}$ are sensitive to particles from new physics theories that could be revealed through the $B_{s}^{0} \to \mu^{+} \mu^{-}$ effective lifetime and the branching fractions of both decays.  New physics processes could influence the $B_{s}^{0} \to \mu^{+} \mu^{-}$  effective lifetime and beanching fraction independently therefore the two observables are complementary in the search for new physics. 


The $B_{s}^{0} \to \mu^{+} \mu^{-}$ effective lifetime is measured, for the first time, at 2.04 $\pm$ 0.44 (stat) $\pm$ 0.05 (syst) ps and the branching fraction $\mathcal{B}(B_{s}^{0} \to \mu^{+} \mu^{-})$ is measured at a statistical significance of 7.9 $\sigma$ to be  (2.8 $/pm$ 0.6) x 10$^{-9}$ where the uncertainty includes both statistical and systematic uncertainties. An upper limit is placed on the branching fraction $\mathcal{B}(B^{0} \to \mu^{+} \mu^{-})$  $<$ 6.1 x 10$^{-10}$ at the 85 $\%$ confidcence level. All results are consistent with the Standard Model predictions. 

