% ************************** Thesis Abstract *****************************
% Use `abstract' as an option in the document class to print only the titlepage and the abstract.
%\begin{abstract}
%This is where you write your abstract ...
%\end{abstract}
\chapter{Abstract}

This dissertation documents studies of very rare B meson decays at the LHCb experiment on data taken during the first two experiment runs of the Large Hadron Collider (LHC).



The LHCb experiment was designed to test the Standard Model of particle physics and search for new physics theories that go beyond the scope of the Standard Model, through the decay of $b$-hadrons produced in high energy proton-proton collisions at the LHC. The measurements described in this dissertation were made using data samples of proton-proton collisions with integrated luminosities of 1.0, 2.0 and 1.4 fb$^{-1}$, collected at centre-of-mass energies of 7, 8 and 13 TeV respectively. %The measurements detailed in this thesis uses 1.0, 2.0 and 1.4 fb$^{-1}$ of data collected by the LHCb experiment in proton-proton collisions at center-of-mass energies of 7, 8 and 13 T$e$V.


The branching fractions and effective lifetimes of the very rare $B_{s}^{0} \to \mu^{+} \mu^{-}$ and $B^{0} \to \mu^{+} \mu^{-}$  decays are sensitive to particles from new physics theories. %The very rare decays $B_{s}^{0} \to \mu^{+} \mu^{-}$ and $B^{0} \to \mu^{+} \mu^{-}$ are sensitive to particles from new physics theories that could be revealed through the $B_{s}^{0} \to \mu^{+} \mu^{-}$ effective lifetime and the branching fractions of both decays.  
New physics processes could influence the $B_{s}^{0} \to \mu^{+} \mu^{-}$  effective lifetime and branching fraction independently, and therefore the two observables are complementary in the search for new physics. 




The $B_{s}^{0} \to \mu^{+} \mu^{-}$ decay is observed with a statistical significance of 7.8$\sigma$ and the branching fraction is measured as  $\mathcal{B}(B_{s}^{0} \to \mu^{+} \mu^{-}) = (3.0 \pm 0.6^{+0.3}_{-0.2}) \times 10^{-9}$. The $B_{s}^{0} \to \mu^{+} \mu^{-}$ effective lifetime is measured for the first time as 2.04 $\pm$ 0.44 (stat) $\pm$ 0.05 (syst) ps.
An upper limit is placed on the branching fraction $\mathcal{B}(B^{0} \to \mu^{+} \mu^{-})< 3.4 \times 10^{-10}$ at the 95 $\%$ confidence level. All results are consistent with the predictions of the Standard Model. 

%The $B_{s}^{0} \to \mu^{+} \mu^{-}$ effective lifetime is measured, for the first time, as 2.04 $\pm$ 0.44 (stat) $\pm$ 0.05 (syst) ps and the branching fraction $\mathcal{B}(B_{s}^{0} \to \mu^{+} \mu^{-})$ is measured at a statistical significance of 7.9 $\sigma$ to be  (2.8 $/pm$ 0.6) x 10$^{-9}$ where the uncertainty includes both statistical and systematic uncertainties. An upper limit is placed on the branching fraction $\mathcal{B}(B^{0} \to \mu^{+} \mu^{-})$  $<$ 6.1 x 10$^{-10}$ at the 95 $\%$ confidcence level. All results are consistent with the Standard Model predictions. 
