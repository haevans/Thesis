\chapter{Summary and Future Outlook}
%Summaries the results and give future predictions

Hmm what to put in here?

Summary of the BF analysis with a contour plot of the BFs? And the summary plot that goes into the paper.

Conclusions about the lifetime, discussion about the size of the uncertainties and the future prospects. Therefore I need nice eversions of the plots that I showed to Patrick. Also I should download the paper and the plots.


lumi    mean    median    std dev
4.4     0.348   0.322     0.134
8.0     0.2238  0.2157    0.0494
50      0.078969 0.0787 0.0057
300    0.03172   0.03171 0.00091


Right, so what do I want to put into my thesis.


The search/testing on the SM has been going on for many years, this has included the search for bs2Mumu and bd2mumu decays which has gone on for over 30 years.
LHCb was designed to measure b decays, Bs2MuMu was outlined in the roadmap of the experiment. First evidence came from LHCb in X with Y, and the first eviddence and first obervation came in the combination paper. With the observations allows measurements to be made of properties such as the EL.
Bs2MuMu decays are important to 
This thesis decouments the first single experiment observation of bs2mumu and the first measurement of the Bs2MuMu effective lifeimte. I should say what the measurements are and also perhaps talk about the ratio of Bs/Bd? Although I'm not sure that is relevant for this but I could inlucde the contour plot as well. First give any/all BF details then move on the the EL. The measured BF allow contraints to be places on NP models, here's a paper that uses the rresults already! 
EL, it's pointless to give A_DG at the moment and therefore a comparison is pointless, but this is non the less a very important measurement, it's a proof of concept that LHCb can do this and sets the scene for the future.

So what does the future hold for the EL measureent? 
Mention the BF measurement that is in the EOL, say the senstivity is supposed to be X in the future as reported in Y at the end of the chapter
The EL, based on the obeerved number of decays, predictions can be makde about the future. However first let us consider the unceratinty we got for the measured number of decays, look it's worse that expected but perfectly reasonable. So for the future, we expect with 8fb, 50 fn and 300 fb. Therefore it will be a long time before it's interesting but with 30fb it will start to be good. However these will be conservative estimates since the analysis stratgu will change as more data is avalible. These are the expected statistical unceratinties and the corresponding systmeatics must decrease too, the largest could are limited by data but some are not, these must be address in the future. An alternative method would be lifetime unbiased which would remove the problems from the acceptance function. 
I would say to put the plots for the current uncertaintues and those at the end of Run 2 and just the numbers for the 50 and 300 fb prediction.
Some nice sentence to finish off. Bs2MuMu has been a decay to watch for years and it still is for the future, the precision of the BF makes it imprnat since NP could still be there and also the Bd is syet to be observed! The ratio is well important. And with more data we get better measurements of A_DG which will ffuther constrain NP.


\section{Summary}
The predictions of SM has been tested lots using many different experiments, the highest energy experiment being the LHC. So far results from the experiments at the LHC have confirmed the predictive power of the SM to describe interactsions at the very small level. The study of the SM includes the study of \bmumu decays which could either reveal new physics effects or place strong constraints on new physics theories.
The search for \bsmumu began over 30 years ago and is continued at the different experiments on the LHC. These decays are useful as indirect seraches for new physics processes that could either reveal new physics effects or place strong constraints on new physics theories. 
The LHCb experiment was decay to test the SM in CP violating decays and rare b hadron decays, \bmumu decays were indicated one of the key measurements to be made at LHCb~\cite{Adeva:2009ny}.
The first evidence for \bsmumu decays was found by the LHCb collaboration with 1.0 and 1.1~\fb of data from $pp$ collisions from Run~1 of the LHC~\cite{Aaji:2012nna}. A combined analysis of the Run~1 data from the CMS and LHCb collaborations produced the first observation of \bsmumu decays and the first evidence for \bdmumu decays~\cite{CMS:2014xfa}. The measured BFs are consistent with the SM predictions and the observation of \bsmumu decays now allows the measurements of properties of these decays to be done. Therefore these decays are still very much interesting. 

%This disseration presented the first single experiment observation of \bsmumu decay and the first measurement of the \bsmumu effective lifetime along with a new upper limit on \bdmumu decays with data collected during Run 1 and Run 2 of the LHC. 
This disseration presented the measurement of the \bsmumu BF and effective lifetime and the search for \bdmumu decays with 4.4~\fb of Run 1 and Run 2 data with the LHCb collaboration. The \bsmumu decay was observed with a statistical significance of 7.8 sigma and the \bdmumu decay with a statistical significance of 1.8? sigma, therefore giving the first single experiment observation of \bsmumu. 
The measured BFs are


Due to the small statistical signaficnace of the \bd mode a limit has been placed on the \BF at XXX. The measure values are consistent with the SM and figure? showns a coparison with previously measured results. However due to the current precision there is still space for NP to appear. Looking ahead the \bd mode is well important and so it the ratio of BFs to test the flavour structure of the SM.

The observation of \bsmumu decays allows other properties of them to be studied, including the \bsmumu effective lifetime. The effective lifetime offers a complementary search for NP, that could revael NP effects that are not necessarily apparent in the BF measurement. The effective lifetime has been measured for the first time with 4.4~\fb of Run 1 and Run 2 data with the LHCb collaboration as

which is consistent with the SM prediction at the 1.0 sigma level. The current measurement leads to a unphysical value of \ADG with a range of XXX at the 1 sigma level. Although the current precision says nothing about new physics there is hope for the future it is a proof of concept measurement.

\section{Outlook}

The measured values of the BF and EL presented here still leave plenty of room for new physics. The LHC is expected to give to following and far into the future we get the hi-lumi lhc.
 Looking in to the furture with the high luminoscity LHC, the LHCb expeciment could achive somthing great with the BF analysis.
ratio of BFs at 40 $\%$ at the end of Run 4 and 20$\%$ with the high luminoscity LHC ~\cite{Aaij:2244311}. Run 4 is 50 fb and HL_LHC is 300 fb

Simiarly the precision of the \el is expected to increase. The expected unceratinies for the observed number of decays in 4.4 fb are shown in figure~\ref{} alongside the expectataios for the end of Run~2. We did slightly worse than we could have done with the data we've got at the moment but by the end of Run 2 the expected precision is .... Looking further in to the future with 50 and 300 fb we expect to get the following, which is good. Howver this is all based on the current analysis method which will change as more data is collected. Therefore the future looks hopeful.

What do I need to find out to do this better?
- the EL predictions ( I think that it'd be easiest to ignore the inverse square law stuff).
