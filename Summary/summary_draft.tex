\chpater{Summary and outlook}%Summary draft
\label{sec:summary}

The LHC has provided a great place to test the SM, it has found the Higgs and but no significant new physics effects have been discovered so far at the LHC. But there have been some anomolies found in rare decays. Therefore everything is still very interesting and Run 2 is very important. 
Soemething also about anomolies in rare decays.

The search for \bsmumu and \bdmumu decays has been going on for over 30 years because these decays offer an excellent place to indirectly search for new physics effects. Since the start of data taking at the LHC the data collected has allowed unpredicdented sensitivity to \bmumu decays. This lead to the first evidence of \bsmumu by LHCb with X \fb in Run~1, and the first observation of \bsmumu and evidence for \bdmumu from the combined analylsis of the full Run~1 data sets from CMS and LHCb. The measured results have, so far, all been consistent with the predictions of the SM and have enabled constraints to be places on BSM theories. %The observation of \bsmumu has now allowed measurements of properties of this decay to be investigated. 


This dissertation documents the first single experiment observation of \bsmumu alongside a upper limit on the \bdmumu branching fraction of
\begin{equation}
%\begin{align}                                                                                         
\begin{split}
  \mathcal{B}(B^{0}_{s} \to \mu^{+} \mu^{-}) &= (3.0 \pm 0.6^{+0.3}_{-0.2}) \times 10^{-9} \\
  \mathcal{B}(B^{0} \to \mu^{+} \mu^{-}) &< 3.4    \times 10^{-10}
%\end{align}                                                                                           
\end{split}
\label{eq:BFresults2}
\end{equation}
The results are combined in one nice plot in Figure~\ref{}. Although the results are consistent with the SM prediction, the current precisions still leaves rooms for new physics effects to be seen in \bsmumu decays. Furthermore in new physics theories, including those with large tan($\beta$), the most probably BF results are consistent with the SM predictions. %Furthermore the ratio of the \bsmumu and \bdmumu branching fractions is still very interesting as a test of the flavour structure of the decays
The expected senstivity of LHCb to the \bsmumu and \bdmumu branching fractions is .....



The observation of \bsmumu decays has enabled the study of the \bsmumu \el. The \el offers a complementary search for new physics effects in \bsmumu decays compared to the branching fraction, new physics effects could be evident in both or just one of these variables. This disserataion presents the first measurement of the \bsmumu \el lifetime at

\begin{equation}
\tau_{\mu\mu} = 2.04 \pm 0.44  \pm 0.05 \text{ ps}
\end{equation}

The central value lies outside the allowed regin but it is consistent with the SM prediction at the 1.0 at $\sigma$ and it is 1.5 $\sigma$ from the lifetime of the light \bs mass eigenstate. The measured value of \tmumu does not allow a meaningful value of \ADG to be evaluated, it gives a range of \ADG values of XX and the X sigma level. However the result is still important at illustrating what the LHCb detector is capable of and with more data the sentivity of this measurement will increase. 

The precision needed to distinguish new phsyics effect using the \bsmumu \el in the extreme case that \tmumu = \tL is 0.038 \ps as discussed in Section~\ref{}. The statisitical uncertainty with 4.4~\fb is far above the necessary precision, however future Runs of the LHC will reduced this uncertainty. Based on the observed number of decays with 4.4~\fb toy studies following the set up in X have been used to find the expected precision with 8, 50 and 300\fb. The uncertainties from the toy studies from the current observed number of decays with 4.4~\fb and the number expected with 8~\fb at the end of Run~2 are shown in Figure~X. The median uncerainty is reduced to 0.2~ps and the spread in the unceratinies is much smaller however this is stil too large to see new phsycis effects. Looking further into the furture, at the end of Run~3 with 50~\fb LHCb could achceive a sensitivity of 0.08 ~\ps and with 300~\fb from the high luminoscity LHC the expected sensitivity is reduced to 0.03~\ps and new physics effect could be reavealed. The mean values and spreads of the data are shown in Figure~\ref{}. Howver the expected precisions are based on the current analysis stratgy which was designed to measure the \bsmumu \el with very low statisitics, as more data is collected it is inevitable that the analysis strategy will change to reflect this therefore the future predictions are probably conservative estimates of what LHCb could acheive. 

The study of \bmumu decays has been very interesting for a long time and will continue to be so. It's clean decay signature makes it siltl a great place to search for new physics effects. The future looks good, because we still need to observe the Bd and the effective lifetime is very important. It will ve vert intersting in the future. Also the Bs2MuMu measurement is dependant on A_DG, therefore it's important to measure with the lifetime.


%I should also probably show some understanding of the wider physics world, prephaps put in some context of rare decays? However this may well become clearer once I have written the intro and theory chapter.

What plots/numbers?
- plot of the measurements over time
- the ratio plot with the BSM theories and our result on it?
- the contour plots overlapped for all the different experiments? ATLAS, CMS+LHCb and LHCb?
- future predictions of the BF? I need to get some details from somewhere
- future predictions of the lifetime, I'm not sure how to include this since we do beat the inverse lumi relationship. I think having the speads are important, prehaps put into a table since the plots don't look great, although the plot of 4.4fb and Run 2 looks ok, however this plot doesn't really add very much. 

I should have a look at the paper, and some of the emails that explain the current importance of the effective lifetime measurement and see what the future predictions are in the EOL but also there should be some predictions for Run 2 and Run 3 somewhere. I should read again what Siim wrote and prehaps Max?
