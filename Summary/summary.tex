\chapter{Summary and Outlook}
\label{sec:summaryandoutlook}

\section{Summary}
The LHCb experiment was built to test the predictions of the Standard Model and search for new physics effects through the study of $\mathcal{CP}$ violating and rare decays of $b$-hadrons. So far measurements performed using the LHCb experiment, and other LHC experiments, show no significant deviations from predictions and confirm the predictive power of the SM. 
The search for \bmumu decays was identified as one of the key measurements to be made with the LHCb experiment~\cite{Adeva:2009ny} as an indirect search for new physics.
In 2011 LHCb joined search for these decays, that began over 30 years ago. using the unprecedented energies available at the LHC. The first evidence for \bsmumu decays was found by the LHCb experiment with 2.1 \fb of Run~1 data from $pp$ collisions~\cite{Aaij:2012nna}. A combined analysis of the Run~1 datasets from the CMS and LHCb experiments produced the first observation of \bsmumu decays and the first evidence for \bdmumu decays~\cite{CMS:2014xfa}. The measured branching fractions of these decays are consistent with the SM predictions and place strong constraints on new physics models but the precision of the measurements still leaves room for new physics effects to be revealed. With the observation of the \bs mode the search for \bsmumu decays is over and properties of these decays, including the effective lifetime, can now be studied. The effective lifetime offers a new observable to test the SM in \bsmumu decays that is complementary to the \BF. 

The measurements of the \bmumu \BF and the \bsmumu effective lifetime with 4.4~\fb of Run~1 and Run~2 data collected by the LHCb experiment were presented in this dissertation. The measured \BFs are
\begin{equation}
%\begin{align}                                                                            
\begin{split}
  \mathcal{B}(B^{0}_{s} \to \mu^{+} \mu^{-}) &= (3.0 \pm 0.6^{+0.3}_{-0.2}) \times 10^{-9\
} \\
  \mathcal{B}(B^{0} \to \mu^{+} \mu^{-}) &= (1.5^{+1.2 +0.2}_{-1.0 -0.1})    \times 10^{-\
10}
%\end{align}                                                                              
\end{split}
\label{eq:BFresults2}
\end{equation}
The \bs mode was observed with a statistical significance of 7.8 $\sigma$, making this result the first single experiment observation of this decay. The \bd mode was observed with a significance of 1.6 $\sigma$, therefore a limit is placed on the \BF of $\mathcal{B}$(\bdmumu)$ < 3.4 \times 10^{-10}$ at the 95$\%$ confidence level. The measured values are concisest with the SM predictions. %plot of Bd bv Bs?? with the others superimposed?


The effective lifetime of \bsmumu decays was measured for the first time to be 
\begin{equation}
\tau_{\mu\mu} = 2.04 \pm 0.44 \pm 0.05 \text{ ps}
\end{equation}
which within 1.0 $\sigma$ of the SM prediction. The measured lifetime is longer than the SM prediction therefore giving an unphysical central value of \ADG, the result is consistent with \ADG = +1 hypothesis at 1.0 $\sigma$ and with \ADG = -1 hypothesis at 1.4 $\sigma$. Although the current precision of the measurement does not enable constraints to be placed on new physics models it is important to illustrate the ability of the LHCb experiment to make this measurement.


\section{Outlook}
The measured values of the branching fractions and the effective lifetime still leave plenty of room for new physics effects to be observed with these decays. At the end of Run~2 of the LHC, the LHCb dataset will have almost doubled to be 8~\fb, enabling the precision of these measurements to be improved. Looking further ahead into the future LHCb is expected to collect 50~\fb of data by the end of Run~4 and with the high luminosity LHC up to 300~\fb could be recorded. 

Not only are the \BF measurements in themselves interesting to test the SM but the ratio of the \BFs of the two modes is also useful to test the SM, in particular the flavour structure of the SM and new physics models. The current precision of the ratio of \BFs is X$\%$, the future Runs of the LHC will enable the precision of the ratio of \BFs to be reduced to 40$\%$ with 50~\fb of $pp$ data and 20$\%$ with 300~\fb of data~\cite{Aaij:2244311}. 

The expected uncertainty achievable by the LHCb experiment for the \el at the end of Run~2 and after future Runs of the LHC has been computed using pseudoexperiments based on the observed numbers of decays with 4.4~\fb and the current measurement strategy. At the end of Run~2 the median uncertainty of the \el will be $\sim$0.2 \ps which is reduced to $\sim$0.08~\ps with 50~\fb and $\sim$0.03~\ps with 300~\fb. Therefore with 300~\fb the precision on the effective lifetime will start to be useful to distinguish new physics effects with this measurement. However the expected uncertainties are conservative estimates because they are based on the current measurement strategy which was designed for low expected statistics therefore the precision of the measurement could be much better as different analysis methods can be taken advantage of with more statistics. 
The current systematic uncertainty is 0.05 \ps which is too large to allow possible new physics effects to be observed however several components contributing to the total, such as the fit accuracy and the acceptance function systematic, will be reduced with the availability of more data enabling greater precision on the measurement. With more data, an alternative analysis approach could be used which would reduce the systematic uncertainties on the \el, the selection criteria can be designed so that it does not bias \bsmumu decay time distribution, therefore removing the need for an acceptance function and the associated systematic.
%Do I want to include more discussion on this?? Or prehaps add as a appendix? I don't think a long discussion will add much here but prehaps the details could be useful or someone will ask about it ....


The study of \bmumu decays has been in progress for over 30 years and with the energy and luminosity available at the LHC, the study of these decays is just as interesting as it ever was. As more data is collected by the LHCb experiment new physics effects will have less and less space to hide, it will either be seen in \bmumu decays or these decays will place ever tighter constraints on new physics models. 
