\chapter{Systematic Uncertainties and Cross Checks}
\label{sec:systematics}
%Follow the analysis note
The measured \bsmumu effective lifetime presented in Chapter~\ref{} is influenced by various systematic biases arising from different areas of the analysis proceedure. In this Chapter the size of systematic biases is estimated and several cross checks are made on the measurement strategy to ensure the uncertaintues quoted on the final results are correct. The total systematic uncerainties for measuring \tmumu and \Gmumu are give at the end of the Chapter.

All the work presented in this Chapter was completed for this Thesis.

\section{Fit Accuracy}
\label{sec:fitaccuracy}
The fit strategy used to measure the \bsmumu effective lifetime was presented in Chapter~\ref{}, the final fit configuration was chosen by optimising two different types of figures of merit;the mean and width of the pull distributions of free parameters in the fit and the expected uncertainties on \tmumu and \Gmumu. The values for these parameters for 10,000 toy studies using the final fit configuration and assuming the Standard Model \bsmumu branching fraction and effective lifetime are given in Table~\ref{}. The same toy set up as described in Section ~\ref{} but in these toy studies only \bsmumu and combinatorial background decays are generated. 

%Insert table here!

Based on these results several aspects of the fit deserve further investigation, this includes the stability of the fit with different \tmumu values, the biased pull distribution for \bsmumu yields and the overall bias in the measured values of \tmumu and \Gmumu. The biased pull distributions for \tmumu were discussed in Section~\ref{}.

\subsection{Input lifetimes for toy studies}
The value of \Gmumu is accurately returned by the fit, as shown in the pull distribution in Figure~\ref{} for the expected number of decays. However, it is necessary to understand if this is due to accurate background subtraction by the sPlot method or it it could be resulting from similarities between the decay time distributions of \bsmumu and combinatorial background decays. As a test, toy studies were performed for a range of generated \bsmumu lifetimes different to the Standard Model prediction. Only \bsmumu and combinatorial background decays were generated in the toy studies so that the small contamination from mis-identified backgrounds does not mask the effects of using different lifetimes. The results of 10,000 toy studies are shown in Table~\ref{} and the different lifetimes all return accurate pull distributions for the fitted \Gmumu values with means and widths consistent with 0 and 1, respectively. Therefore the fit returns an accurate measured value for a range of \bsmumu lifetimes.

\subsection{\bsmumu yields}
The pull distributions for the \bsmumu yields in the mass fit have biased mean values of 0.009 \ps (?) as shown in Table(s)~\ref{}, implying that the mass fit does not accurately estimate the \bsmumu yield. However the pull distribution for \Gmumu is accurate therefore this bias in the \bsmumu yield could originate from a different source.

In the toy studies the number of expected decays, $N_{expt}$, in the mass range 4900 - 6000 \mevcc are given in Table~\ref{}. These numbers are used as the basis for the toy studies however the number of decays generated is fluctuated for each study about the expected value using a Poisson distribution. This enables an extended maximum likelihood fit, where the total number of event is a free parameter, to be used to fit the mass distribution. To acheive an accurate pull distribution of the measured \bsmumu yields from the mass fit the uncertainties on the measured yields must be distributed according to a Gaussian funtion. This will be true in the high statistics limit where a Poissioan distribution si a good approximation of a Gaussian distribution. However the current data avaliable does not contain high statistics for \bsmumu decays therefore the uncertainty on the measured yields is proportional to $\sqrt{N_{expt}}$ and does not have a Gaussian distribution. This effect would shift the mean value of the pull distribution but not lead to an incorrect extimation of the \bsmumu yield. The `fractional bias', $N_{meas} - N_{expt}/N_{expt}$, where $N_{meas}$ is the measured \bsmumu yield is shown in Figure~\ref{} and supports this explaination by producing a mean consistent (?) with zero, with a bias of 0.5 $\%$. Furthermore, the pull distributions for \bsmumu yields for toy studies with higher statistics have means that move towards 0 as shown in Figure~\ref{} for 50 and 300 \fb. 

Therefore the mass fit returns accurate yields for \bsmumu and the biased pull distributon arises from the low statitics of the data set. The same reasoning can be applied to the pull distribution of the yields of combinatorial background decays that have a slighty less biase mean value of 0.007 compared to the \bsmumu yields.

\subsection{Overall bias on \tmumu and \Gmumu}
The remaining area of the fit to investigate is any underlying bias in the fit on the measured values of \tmumu and \Gmumu. (As discussed in Section~\ref{} the pull distribution for the measured effective lifetime is biased for the expected number of statistics but the pul distribution for \Gmumu produces a mean and width consistent with 0 and 1, respectively. However the coverage of the unceratinies of both \mmumu and \Gmumu are reasonable and the biased \tmumu pull arises from the likelihood function as discussed in Section~\ref{}. 

The overall bias in the fit for measuring \tmumu and \Gmumu is evaluated from the difference between the measured and generated values from toy studies. A total of 10,000 toy studies are performed generating only \bsmumu and combinatorial background decays so the fit bias is not masked by contamination from mis-identified backgrounds.  The bias arising from the small contribution of mis-identified backgrounds in the mass range of the fit is evalutated in Section~\ref{}. The difference between the measured and generated \tmumu and \Gmumu values si evaluated for toy studies for the expected and observed number of \bsmumu decays. The fit bias is evaluated for the observed number of decays because these are fewer than expected. The resulting distributions are shown in Figure~\ref{}. The mean of the \Gmumu difference distribution is consistent with 0 and the mean of the \tmumu difference distribution is 0.03 \ps for the observed number of decays.  Therefore giving a systematic uncertainty of 0.03ps for the fit accuracy of \tmumu and no systematic uncertainty for \Gmumu.

\section{Background contamination}
\label{sec:BKGcontaim}
The mass fit configuration used to measure the \bsmumu effective lifetime only includes components for \bsmumu and combinatorial background decays and only candidates with a \bs mass between 5320 - 6000 \mevcc are used in the mass fit. Although the majority of background decats from mis-identified semi-leptonic and \bhh decays and \bdmumu decays fall outside this mass window, as shown in Figure~\ref{} the tails of some backgrounds still end up in the mass window. The backgrounds that need to be considered are \bdmumu, \bhh, \lambdab, \bdpimunu, \bsKmunu, \bupimumu, \bdpimumu, \bcjpsimunu, with \bhh, \bdmumu and \lambda being of particular importance. 

The number of expected background decays and their mass \pdfs in the mass range 4900 - 6000 \mevcc were computed using the methods described in Chapter~\ref{}. The number of decays from each background type expected in the smaller mass range 5320 - 6000 \mevcc are computed by integrating the mass \pdfs. The expected yields for each background in both mass ranges are given in Table~\ref{}. The expected yields are all $<$ 1 for each background source. 

The impact of backgrounds not modelled in the mass fit on the measured lifetime and inverse lifetime is evaluated using two sets of toy studies. The toy studies have the same general set up as described in Section~\ref{}. One set of toy studies assumes there are no background other than the combinatorial background and therefore only \bsmumu and combinatorial background candidates are generated. The second set of toy studies generates all possible background decays. The expected yields are fluctuated using a Poissonian distribution around their expected values to 1 decimal place. For each configuration 10,000 toy studies were performed and the pull distributions for \Gmumu of each toy set up is compared. The pull distributions for \tmumu arenot used due to their non-Gaussian distribution disussed in Section~\ref{}. 

The inclusion of all the background decays causes a shift in the mean of the \Gmumu pull distribution of 0.025 \ps$^{-1}$. Therefore, assuming the {\it expected uncertainties} in Section~\ref{} of X and Y the systematic shift from not including all backgrounds in the fit configuration is Y for \tmumu and Y for \Gmumu. 

The expected number of \bdmumu decays assumes the Standard Model branching fraction, however the current world average (excluding the results presented in Chapter~\ref{}) is higher than the SM prediction by $\sim 3$. The tpy studies were repeated with the world average but the shift in the mean of the pull distribution was smaller, therefore the larger value from the SM predictions are used. {\it why is this smaller??}

\subsection{Mass \pdf parameters}
\label{sec:massPDFsyst}
The data collected in Run~1 and Run~2 are combined for the measurement of the \bsmumu effective lifetime and the mass and decay time fits are applied to the combined data. However the parameters used in the mass \pdf in Table~\ref{} were evaluated specifically for Run~1 data and different parameters are avaliable for Run~2 data. Therefore the influence of the choice of mass \pdf parameters on the measured \tmumu and \Gmumu values must be evaluated. 

Toy studies are performed to understand the size of the impact of the mass fit choice on the effective lifetime measurement. Only \bsmumu and combinatorial background decays are generated to seperate mass \pdf effects from the contamination of mis-identified backgrounds and \bdmumu decays in the mass window. \bsmumu candidates are generated using the Run~1 parameters in Table~\ref{} but the mass fit is performed using the Run~2 parameters in Table~\ref{}. The pull distribution for 10,000 toy studies of \Gmumu from this configuration are compared with those from toy studies where Run~1 parameters are used to generate and fit the mass distribution. The change in the measured lifetime with the mass \pdf parameters is negliable as shown in Figure~\ref{} that overalys the pull distributions for the two toy configurations studied. Therefore no systematic unceratinty is assigend. 

\subsection{Acceptance function accuracy}
\label{sec:accptsyst}
The decay tie acceptance comes from weighted simulated decays. This relies on the assumption that the weighted simualted decays model the data reasonable well. To test this assumption, as well as the stratgy used to measure the \bsmumu effective lifetime, the lifetime of the more abundant \bdkpi and \bskk decays are measured.

\bdkpi decays have a much larger branching fraction the \bsmumu decays at XXX and are therefore more abundant in data. The selection requirements used to identify these decays in 2011, 2012, 2015 and 2016 data are detailed in Chapter~\ref{} and are kept very similar to the \bsmumu selection. All candidates are required to be TIS, although this considerably reduces the statistics, \bhh triggers are $\%$ efficient, the TIS triggers are relatively unbiased with respect to the decay time similarly to triggers that select \bsmumu decays. Whereas TOS triggers that identify \bhh decays create a large bias on the decay time distribution due to the dependance on the trigger lines on IP, IP \chisqd and flight distance variables. The DLL$_{K\pi}$ variable is used to seperate \bdkpi decays from other \bhh decays, as detailed in Section~\ref{} and candidates are reconstructed with the daughter with the highest DLL$_{K\pi}$ value assigend the kaon mass hypothesis and the daughter particle with the lowest DLL$_{K\pi}$ value the pion mass hypothesis.

The measurement of the \bdki lifetime is performed in the same way as the \bsmumu effective lifetime measurement and all years of data are combined into one data set. An extended, unbinned maximum likelihood fit is perfomred to the \bdkpi mass distribution. Components for \bdkpi, \bskpi and combinatorial background decays are included in the mass fit. Both B meson decays are modelled by Crystal Ball functions with parameters taken from some source. Combinatorial background decays are modelled by an exponential function with the slope left free in the mass fit. The mass fit, shown in Figure~\ref{}, is used to calculat sWeighted that are re-normalised using equation~\ref{}. 
