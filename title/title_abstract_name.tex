\begin{titlepage}
%  \centering
%  \includegraphics[width=0.25\textwidth]{University_Crest.pdf}\par\vspace{1cm}
%  {\scshape\LARGE  University of Cambridge \par}
%  \vspace{1cm}
  %{\scshape\Large PhD Thesis\par}
%  \vspace{1.5cm}
 
\begin{center}
 {\LARGE\bfseries Measurements of \boldmath{$B \to \mu^{+} \mu^{-}$} decays using the LHCb Experiment \par}
  \vspace{1cm}
  {\Large\itshape Hannah Mary Evans}
\end{center}

  \vspace{2cm}


\noindent {\Large \bfseries Abstract}
  \vspace{0.5cm}

\normalsize
This dissertation documents a study of very rare $B$-meson decays at the LHCb experiment, using data taken during the first experiment run ofthe Large Hadron Collider (LHC) and during the second experiment run until September 2016.



The LHCb experiment was designed to test the Standard Model of particle physics and to search for New Physics effects that go beyond the scope of the Standard Model through the decay of $b$ hadrons produced in high energy proton-proton collisions at the LHC. The measurements described in this dissertation are made using data samples of proton-proton collisions with integrated luminosities of 1.0, 2.0 and 1.4~fb$^{-1}$, collected at centre-of-mass energies of 7, 8 and 13~\tev, respectively. %The measurements detailed in this thesis uses 1.0, 2.0 and 1.4 fb$^{-1}$ of data collected by the LHCb experiment in proton-proton collisions at center-of-mass energies of 7, 8 and 13 T$e$V.                    

The branching fractions of the very rare \bdmumu and \bsmumu decays and the effective lifetime of \bsmumu decays are precisely predicted by the Standard Model and are sensitive to effects from New Physics.
%The branching fractions and effective lifetimes of the very rare $B_{s}^{0} \to \mu^{+} \mu^{-}$ and $B^{0} \to \mu^{+} \mu^{-}$  decays are sensitive to particles from new physics theories. %The very rare decays $B_{s}^{0} \to \mu^{+} \mu^{-}$ and $B^{0} \to \mu^{+} \mu^{-}$ are sensitive to particles from new physics theories that could be revealed through the $B_{s}^{0} \to \mu^{+} \mu^{-}$ effective lifetime and the branching fractions of both decays.                                                                                                         
New Physics processes could influence the $B_{s}^{0} \to \mu^{+} \mu^{-}$  branching fraction and effective lifetime independently, and therefore the two observables are complementary. %in the search for New Physics.                                                                   




The $B_{s}^{0} \to \mu^{+} \mu^{-}$ decay is observed with a statistical significance of 7.8$\sigma$ and the branching fraction is measured to be $\mathcal{B}(B_{s}^{0} \to \mu^{+} \mu^{-}) = (3.0 \pm 0.6^{ +0.3}_{ -0.2}) \times 10^{-9}$. The $B_{s}^{0} \to \mu^{+} \mu^{-}$ effective lifetime is measured for the first time as 2.04 $\pm$ 0.44 $\pm$ 0.05~\ps.
The \bdmumu \BF is measured as $\mathcal{B}(B^{0} \to \mu^{+} \mu^{-}) = (1.5^{+1.2}_{-1.0}^{+0.2}_{-0.1})\times 10^{-10}$ with a statistical significance of 1.6$\sigma$. An upper limit is set for the \BF of $\mathcal{B}(B^{0} \to \mu^{+} \mu^{-})< 3.4 \times 10^{-10}$ at the 95$\%$ confidence level. All results are consistent with the predictions of the Standard Model.                                                   




%  \vfill
%  {\large
%  Supervisor  \\
%  Prof. Valerie Gibson
%  %Dr Marc Olivier Bettler \\
%  %Dr Harry Cliff \\
%  }
%  \vfill%

% Bottom of the page
%  {\large This dissertation is submitted for the degree of Doctor of Philosophy \\
\\
%June 2017}
\end{titlepage}
