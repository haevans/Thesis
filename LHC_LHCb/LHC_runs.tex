5 December September 2016: End of 2016 data taking period.
The 2016 data taking period ended this morning. This was a very successful year, both because of the excellent performance of the accelerator and of the LHCb experiment itself.
The left and central images below show the growth in integrated luminosity during different years of LHC operation. Integrated luminosity is a measure of the number of proton-proton collisions produced by the accelerator. The increase of integrated luminosity with time in 2016 was similar to that in 2012; the period of data taking was, however, shorter. This excellent performance is mainly due to the higher efficiency of the accelerator this year. LHCb has recorded 1.67 fb-1 of proton-proton collisions this year, 5 times more than in 2015, a year in which LHC collider was setting-up its operation at the record energy of 13 TeV. The total integrated luminosity recorded during run 2 reached now the target of 2 fb-1 compared to 3 fb-1 collected in run 1.
   
The number of collected events per unit time for a given physics process, N, is proportional to the instantaneous luminosity, L, which is a measure of the brightness of the colliding beams, and to the characteristic property of interaction, cross-section, σ; N=σL. The cross-section for b- and b-quark production at 13 TeV proton-proton collision is about twice that of run 1 (see 5 August 2016 news, item (1)). Since the integrated luminosity is merely the instantaneous luminosity added up over the time the accelerator is operating, the 2 fb-1 Run 2 data sample contains a larger number of beauty particle decays than the 3 fb -1 Run 1 data sample. The increase of number of recorded charmed particles per unit of integrated luminosity was even higher, by a factor of 5, both due to the higher cross-section at 13 TeV and to improvements in the event selection (so-called “trigger”) during data taking. This is excellent news for future LHCb physics analyses. The right image above shows a typical proton-proton collision event.
