\chapter{CERN, the LHC and LHCb}
\label{CERN_LHC_LHCb}


The European Organisation for Nuclear Research (CERN) was founded in 1954 and began with 12 member states as a organisation to encourage European collaboration and the study of nuclear physics. Since it's foundation the collaborative nature of CERN allowed for large-scale expensive experiments and machines to be built. The Proton Synchrotron was CERN's flagship accelerator, operational in 1959 it had a circumference of 628~m and accelerated protons to XXX~GeV, the highest energy at that time. Now 62 years since it's foundation CERN has grown to include 21 member states \footnote{about the other types of countries involved.} and is still at the forefront of high energy physics research. CERN’s latest accelerator, the Large Hadron Collider (LHC), is most energetic particle accelerator ever built, with a 27km circumference the LHC was designed to protons at~14 TeV. \textit{(make sure to mention ions in the LHC section and explain how their energy is limited by the magnets which get protons to 14 TeV in the LEP tunnel). }  This chapter shall discuss the LHC and the LHC beauty experiment, one of the experiments that uses collisions provided by the LHC.

Not sure about the above but;
%%%%%%%%%%%%%%%%%%%%%%%%%%%%
Here there will be some words introducing this Chapter, I think they should be focused on what makes a nice clear introduction not about the cool facts that I have learnt. I think that this part should be done last and should include a little information about the LHC and LHCb and then explain what this chapter is about. My thesis is about data from the LHCb experiment no about how cool CERN is.

Prehaps add in above where the sources where I got most of the information are from!

\section{The LHC}
\label{LHC}

Rough ideas
%%%%%%%%%%%%%%%

The LHC was designed to cllide protons at 14 TeV. It collides lead-ions too but these are irrelevant to my thesis. The LHC accelterated and collides protont and ipons and it has 2 beams that go in oposite directions.


It starts with protons from hydrogen whccih are accelerated through a chain of past CERN accelerators adapted to feed the LHC. 

The protons are around 450 GeV when they are put into the LHC. They are delivered in bunches. THe LHC then accelerates these beams of proton bunches using RF cavities, bendin the beam around it's ring with dipole magents up to the desirerd energu. Once they have made it, the numches are quite spread out so need to be focued sing quadroploe magnets before they are collided at the 4 intereaction points. Prehaps some details about fills of the LHC? Look here is a picture of the accelerator chain that feeds the LHC. 

As well as the energy as an important charteristic of a collider - because the cross-section to produce various particles increases with energy, the luminoscity  is also well important. The instantaneous luminoscity is given here and the parameters mean some interesting things. The luminoscity is proportianol to the collision rate. The parameters can be controlled by the magents to suit the different needs of the experiments. For example ATLAS and CMS use all possible luminoscity which decreases with a fill, where as LHCb operates at a lower luminoscity which is made constant through adjustments to the bunch shapes during a fill. Look here is a cool graph that shows all the things.

There are 7 experiemnts on the LHC that make used of these collisions. CMS and ATLAS are general purpose detectors which were designed to search for the Higgs and identify NP particles which are not in the SM. ALICE uses heavy ion collisions to study quark-gluon plasma. LHCf, TOTEM and MODAL fo some other things. The final experiment LHCb collects data that is used in this thesis and will be explained soon.

The LHC began properly in 2010 and ran in 2012 taking data at 7TeV and 8TeV, there was then a long shut down and various things were done. In 2015 data taking started up again, and is still on going, at the higher energy of 13TeV. The nice table here shows the energy the LHC opereated at each year and the integrated luminoscity collected by LHCb. (Not sure if this should go here or somewhere else, leave here for now).



So overall this plan seems to be;
%%%%%%%%%%%%%%%%%%%%%%%%%%%%%%%%%
1> The LHC is a synchrotron to collide protons at 14 TeV (and ions)
2> This is how the LHC gets it's protons and accelerates them (maybe mention the bunch spacing too?)
3> Now only is energy important but so is luminoscity and here is how we get that
4> Here are the experiments that are on the LHC - see they are different
5> This is what the LHC has done since 2010 and here are the numbers of energy and luminoscity for LHCb 


So lets now accumulate some facts about each of those headings and start to make some sentences;
%%%%%%%%%%%%%%%%%%%%%%%%%%%%%%%%%%%%%%%%%%%%%%%%%%%%%%%%%%%%%%%%%%%%%%%%%%%%%%%%%%%%%%%%%%%%%%%%
1> The LHC is a synchrotron to collide protons at 14 TeV (and ions)
a synchrotron 
The LHC was conceived in 1980.
It was built in the LEP tunnel with a circumference of 27km. (To operate at 14TeV is needed 8.3T magnets to bend the beams.)
The LHC can also collide lead ions can get to an energy of 2.76 TeV per nucleon due to the 8.3T magnets.
Protons travel in oposite directions around the ring in seperate vacuum chambers and are bent by dipole magnets.
Protons are brought to collide in 4 seperate places.
The products of these collisions are recoreded by some experiments.

Sentences
%--------
The LHC is a proton synchrotron that was designed to accelerate and collide two beams of protons travelling in oposite directions up to a center-of-mass energy of 14~TeV. Although operation of the LHC began in 2010 it is yet to reach design energy. The purpose of the LHC is to provide high energy proton collisions, the products of which are used for precision tests of the Standard Model (SM) and to search for new physics particles that go beyond the scope of the SM. There are four interaction points on the LHC ring where the beams are brought to collide, at these points various experiments detect and study the products of these collisions. The LHC can also accelerate lead-nuclei up to an energy of 2.76~TeV per nucleon, but it is only the products from proton collisions that are the topic of this thesis.


The protons for the LHC are supplied by hydrogen .... Bring on point 2.
1>


\section{The LHCb Experiment}
\label{LHCb}
