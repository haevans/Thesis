\section{LHCb Experiment}

The LHC experiment was designed to make precise measurements of processes predicted with the SM to indirectly search for hines ot New Physics processes. %Simon has a reference here.
In particular it was designed to measure both CP violating decays and race decays originating from $b$-hadrons. The physics motivations for the LHCb experiment are quite different compared to those the the general purpose detectors, ATLAS and CMS at the LHC, these differences are reflected in the experimental design. The LHCb experiment is a single-arm forward spectrometer that covers the angular region 10 to 250 mrad in the vertical direction and 10 to 300 mrad in the horisontal direction. The central acceptance is determined by the beampipe. The reduced abgular acceptance of LHC compare to the 4\pi acceptance of the GPDs is due to the prcesses that dominate \bbbar pair production in proton collisions. At the LHC \bbbar pairs are mainly produced through quark anti-quark annihilation, gluon - gluon fusion and gluon splitting, the Feyman diagrams for these processes are shown in Fig. XX. %there are references in some theses for this
These production processes lead to the \bbbar pairs being produced at small angles relative to the beampipe. The angular distribution of \bbbar pairs produced is shown in Fig. XX. Therefore although LHCb has a smaller angular acceptance compared to teh GPDs it accepts approximately 40\% of $b$ quarks produced at the LHC.

The side view of the LHCb detector is shown in Fig. X. The coordinate systems has it's origin at the interatcion point where the protons collide on the left hand sige of the diagram, the $z$- axis follow the beam pipe and the y and x ases are the horizontal and vertical directions. 

The detector is composed of a series of sub detectors, starting with the VELO next to the interataction point and ending with the muon stations, as the distance from the interataction point increases so does the size of the sub detetors in order to cover the required angular acceptance.

The sub detectors can be classified in two main categories;
\begin{itemize}
\item {\bf Tracking}, (VELO, magnet and tracking stations) records the trajectories of charged particles through the detector
\item {\bf Particle identification}; (RICH, calorimeters and muon stations) identifies the type of each particle transversing the detector
\end{itemize}

In a typical \bsmumu decay, the \bs will be produced at the interaction point and decay whilst still in the VELO. The muons will then travel the length of the detector to the muon stations. Figure X shows and example of what happens in in a \bsmumu decay.

The following sections describe the tracking and praticle identification sub-detectors in detail, then outline the trigger and some other stuff that is relevant to data collection and analysis at the LHCb experiment. 
