\chapter{Preface}

In 2013 I started my PhD at the University of Cambridge and now, almost four years later, it is finally coming to an end. This dissertation describes the research I have undertaken during those four years; studying $B \to \mu \mu$ decays with the LHCb experiment. Two different measurements of these decays are presented; the measurement of \bdmumu and \bsmumu \BFs and the measurement of the \bsmumu \el. These results have been published in reference~\cite{Aaij:2017vad} and are the first measurements of these decays using data collected during the second experiment run of the Large Hadron Collider (LHC). My contributions have predominately been to the \el measurement, however there is a significant overlap of the two analysis strategies therefore both measurements are described in this dissertation.

A short introduction to particle physics is presented in Chapter~\ref{sec:intro} along with a summary of the past experimental searches and recent measurements of \bmumu decays\footnote{Throughout this dissertation \bmumu refers to both the particle and anti-particle decays of \bd and \bs mesons into two oppositely charged muons}. The theoretical motivation for measuring the \bmumu \BFs and the \bsmumu \el is discussed in Chapter~\ref{sec:theory_chptr}.
%The first three chapters set the scene for measurements described in the later chapters. Firstly, there is a short introduction to particle physics and current measurements of \bmumu decays, then the theoretical motivation for studying these decays is described in Chapter~\ref{sec:theory_chptr}. 
%The research documented in this dissertation uses the Large Hadron Collider and the LHCb experiment which are described in the Chapter~\ref{CERN_LHC_LHCb}. The work presented in these chapters is not my own but summaries the work performed by others and the citations show where credit is due.
Chapter~\ref{CERN_LHC_LHCb} describes the LHC and the LHCb experiment, which provided the data used in the measurements for this dissertation.
%The study of \bmumu decays has been undertaken using data from the LHCb experiment since it began operation, the latest measurement of the \bmumu \BFs and the first measurement of the \bsmumu \el are described in the remaining chapters.
The criteria used to identify \bmumu decays in LHCb data are described in Chapter~\ref{selection_chapter}. This work was carried out over several years by many members, including myself, of the \bmumu LHCb analysis group. My contributions are the study of the `stripping' selection described in Sections~\ref{strippingold} and~\ref{strippingstudies} and the criteria used to identify \bsmumu decays for the \el measurement in Section~\ref{sec:ELsel}.

The measurements of the \bmumu \BFs are described in Chapter~\ref{sec:BFanalysis}; this work was performed by members of the \bmumu LHCb analysis group and the description focuses in more detail on the parts of the analysis strategy that are also used for the \el measurement. My contributions include the technical aspects of this measurement; producing the ROOT ntuples containing data and simulated decays and maintaining the stripping selection applied to data. 
The measurement of the \bsmumu \el and the systematic uncertainties associated with this measurement are described in Chapters~\ref{sec:lifetimemeasurement} and~\ref{sec:systematics}. 
The work documented in these chapters is the result of my own efforts, although it uses inputs from the \BF measurements; the mass shapes for signal and background decays, the yields of \bsjpsiphi decays in data and the expected yields of signal and background decays in data. %The results for the \bmumu \BF and \bsmumu \el lifetime results are published in the paper~\cite{Aaij:2017vad}.

Finally, a summary is given in Chapter~\ref{sec:summaryandoutlook} of the main results documented in this dissertation and the future prospects for the \BF and \el measurements are also discussed. 

%I am grateful to have been a member of the LHCb collaboration over the last 4 years and to have been fortunate to work on the study of \bmumu decays, one of the LHCb experiment's flagship analyses. Completing my PhD has been an challenging experience and I am extremely appreciative of the opportunities being at the University of Cambridge has offered outside of my degree. My time spent rowing with Cambridge University Women's Boat Club included some of the most challenging and rewarding experiences of my life and Cambridge University Cycling Club that sports other than rowing can be very enjoyable. %showed me that sport can continue after rowing and perhaps be even more fun. 
%Apart from sport, I have found great enjoyment in physics outreach events and undergraduate teaching, particularly seeing others become excited about particle physics or understand new concepts. My year spent at CERN also showed me what it is like to work at an international center of research and that the UK has very small mountains. I am glad that my PhD is finally finished, although this dissertation documents the research outcomes of my PhD it does not do justice to all that has gone into my time at Cambridge.
